\documentclass{report}
\usepackage{a4}
%%%%%%%%%%%%%%%%%%%%%%%%%%%%%%%%%%%%%%%%%%%%%%%%%%%%%%%%%%%%%%%%%%%%%%%%%%% 
%  19.10.93 Original   By: ACRM
%  21.10.93 Various additions & corrections; added examples
%  25.10.93 Extra info on ECHO & corrected residue numbering info.
%  27.10.93 Changed structure for docs on whole of AbM
%  11.11.93 Corrected order of global commands and explained required
%           ordering.
%  23.11.93 Added info on ILE CD
%%%%%%%%%%%%%%%%%%%%%%%%%%%%%%%%%%%%%%%%%%%%%%%%%%%%%%%%%%%%%%%%%%%%%%%%%%%
\newcommand{\cs}{CSearch}
\newcommand{\abm}{AbM}
\newcommand{\degree}{\mbox{${}^{\circ}$}}
\newcommand{\localmachine}{local}
\newcommand{\site}{UCL}
\newcommand{\localexe}{{\tt /usr/local/bin}}
\newcommand{\localdat}{{\tt ../data}}
\newcommand{\es}{\relax}
\newcommand{\ang}{\AA}
\begin{document}
\chapter{CSearch: CHARMm Free Congen}

%%%%%%%%%%%%%%%%%%%%%%%%%%%%%%%%%%%%%%%%%%%%%%%%%%%%%%%%%%%%%%%%%%%%%%%%%%%
\section{Introduction}
\cs\ is a version of the conformational searching program, CONGEN, 
written by Dr.\ Robert E.\ Bruccoleri of Bristol Myers 
Squibb\cite{bruc:congen}.
CONGEN was written as an extension to the CHARMm molecular mechanics
package\cite{brooks:charmm} developed by Dr.\ Bruccoleri and other 
workers at Harvard
University under the direction of Professor Martin Karplus.

When Oxford Molecular, Ltd.\ wished to produce a commercial version (\abm)
of the antibody modelling package\cite{martin:loops,martin:thesis}
developed by Dr.\ Andrew Martin
in Professor Tony Rees's group in Oxford, they needed to include
CONGEN as part of the package. However, for licensing reasons
(CHARMm is available as a commercial package from Polygen, Inc.)
they were unable to redistribute the original CONGEN package which
included the complete implementation of CHARMm. Thus \cs, a CHARMm
free version of CONGEN was developed by Dr.\ Martin for OML.

The development of \cs\ was wholly backed by Dr.\ Bruccoleri who
wishes to make his work freely available to as many people as
possible. \cs\ is essentially a CHARMm-free version of CONGEN,
i.e.\ it takes all the source code from CONGEN which is not
part of CHARMm and supplements this with additional code to replace
the required functionality from CHARMm.

The conformational search part of CONGEN relies on CHARMm in the
following ways:
\begin{itemize}
\item Data structures such as molecular topology information and
coordinates,
\item Energy parameters to create the CHARMm potential,
\item Energy calculation and minimisation,
\item Command parsing.
\end{itemize}

In \cs, the CHARMm free version of CONGEN, these issues have been addressed
as follows:
\begin{description}
\item[Data structures] The internal data structures are maintained in
the form used by CHARMm as they were in CONGEN. However, the code which 
fills in these data structures has been completely replaced.
\item[Energy parameters] The parameters which represent the CHARMm force
field have been published in the literature and are thus freely useable.
The original CHARMm code which reads these data has been completely replaced.
\item[Energy calculation] CONGEN evaluates the energy of each sidechain
conformation it generates and checks for Van der Waals clashes using 
routines separate from those in the rest of
CHARMm; these routines have thus been retained in \cs. 
CONGEN also allows
you to apply CHARMm energy minimisation to each conformation as it is 
generated; this option is not available in \cs. The \abm\ package allows
this option to be implemented in Eureka.
\item[Command parsing] The CHARMm command parser has been completely
replaced by a less sophisticated, but much simpler keyword parser.
\end{description}

Normally, when running the \abm\ package, you will never need to worry
about creating input files for \cs. However, you may find it desirable to
explore the conformational space in particular regions of the molecule
without making use of the rest of the \abm\ package. This guide explains the
format of the \cs\ input file allowing you to perform these types of
experiments.

%%%%%%%%%%%%%%%%%%%%%%%%%%%%%%%%%%%%%%%%%%%%%%%%%%%%%%%%%%%%%%%%%%%%%%%%%%%
\section{Understanding CONGEN and \cs}
\label{sec:understanding}
The conformational search procedure is essentially a simple one; a fragment
of protein structure is deleted and rebuilt by spinning around torsion
angles using a grid of, for example, 30\degree. As each new atom
position is generated, this is used as a starting point for spinning the
next torsion angle.

As additional torsion angles are added to the search, the number of 
conformations
generated and the amount of computer time required rises exponentially.
CONGEN implements a number of strategies to reduce the amount of time
required, though searches of more than 5 amino acids will still take
many hours on a machine such as a Silicon Graphics Iris workstation.
Typically a 5 amino acid fragment will take between 30 and 60 minutes. The
exact time depends on the nature of the amino acids present and on the
conformational flexibility available to the peptide.

The conformational search is organised as a tree search. 
The first amino acid whose conformation is searched forms the base of the
tree. Each conformation generated for this amino acid represents a branch
eminating from the tree's main trunk. In turn, each of these branches
(a conformation for the first residue being constructed) splits into a
number of sub-branches as conformations for a second amino acid are
generated using the first amino acid as a starting point. This
sub-branching of the tree continues as each additional amino acid
is added to the search. At the end of each branching system a leaf
represents a fully built conformation for the peptide fragment 
being constructed.

Clearly, it is advantageous if non-productive branches (i.e. those
which will never generate a leaf because they are going to be of very 
high energy, or where it is impossible
to span the gap in the protein structure because the residues are
pointing away from eachother) are pruned as early as possible. 
If branches which are not going to produce any leaves are identified close to 
the root, the program will spend less time generating futile combinations
of residue conformations.
CONGEN implements a number of optimisation strategies in order to
introduce this pruning.

Firstly, the $\phi$ and $\psi$ torsion angles are combined into a
single degree of freedom. Ramachandran style energy maps containing
energies for each $\phi$/$\psi$ combination are stored and only those
combinations with energies below a specified point on the map are allowed.
Three such maps are stored; one for glycine, one for proline and one for
alanine which is used as being representative of all the other amino
acids.

Secondly, as each position is constructed, a distance check is made to
ensure that it is possible to span the remaining gap using the number
of bonds which remain unconstructed. For the purpose of this checking,
a linear peptide is assumed with each bond angle stretched by 5\degree.
If it is not possible to span
the remaining gap using such a stretched linear peptide, this 
$\phi$/$\psi$ angle combination is rejected and the next combination
is examined. Once a $\phi$/$\psi$
combination has been rejected, it becomes unnecessary to build any
of the other residues using this conformation of the current residue
as a base on which to work.

Thirdly, conformations with van der Waals energies higher than a 
specified cutoff are rejected.

Fourthly, the last three residues are constructed using an analytical
procedure developed by G\={o} and Scheraga\cite{go:closure,bruc:closure}. 
This maps the 6 torsional
degrees of freedom available for 3 residues onto the 3 translational and
3 rotational degrees of freedom required to move from one bond vector
to another.

Sidechains are built as the last level in the branching procedure.
CONGEN implements 5 algorithms for generating sidechains\cite{bruc:congen},
 but in
virtually all cases the `ITER' algorithm is used. This proceeds as
follows:
\begin{itemize}
\item Nested iterations over the sidechain torsion angles are performed
until the first energetically acceptable set of sidechain conformations
is generated.
\item All possible conformations for the first amino acid are then
generated and the conformation with the lowest energy is selected and
its energy is recorded.
\item Using this conformation for the first amino acid, the generation of 
all possible conformations is repeated for each amino acid in turn until
all sidechains have been searched. In each case the lowest energy conformation
for each sidechain is retained.
\item The process then iterates from the first sidechain and continues
until the energy converges or an iteration limit is reached.
\end{itemize}

This method for constructing the sidechains produces only one energetically
favourable sidechain conformation per backbone conformation. While the
procedure can be biased by the starting conformation if the number of
iterations required is large, this has been shown to be the most
effective method and the loss of accuracy is small compared with the 
savings in computer time which result from not generating {\em every\/}
possible set of sidechain conformations\cite{bruc:congen,bruc:hy5}.


%%%%%%%%%%%%%%%%%%%%%%%%%%%%%%%%%%%%%%%%%%%%%%%%%%%%%%%%%%%%%%%%%%%%%%%%%%%
\section{Running \cs}
\cs\ is run simply by typing the command:
\begin{verbatim}
   csearch <input file> <output file>
\end{verbatim}
where \verb1<input file>1 is a control file as described below and 
\verb1<output file>1 is a listing file which gives details of the \cs\ run.
In the \abm\ system on \localmachine\ at \site, \cs\ is store in the 
\localexe\ directory.

You should be aware that \cs\ runs can take many hours to complete. The
amount of time required depends on the number of residues being constructed,
the nature of those residues and the conformational flexibility available 
to the loop. Therefore, you should probably run \cs\ as a background task
at lower than normal priority or as a batch job on those machines which
support this type of operation.

%%%%%%%%%%%%%%%%%%%%%%%%%%%%%%%%%%%%%%%%%%%%%%%%%%%%%%%%%%%%%%%%%%%%%%%%%%%
\section{The \cs\ Input File}
The \cs\ input file consists of keywords followed, where necessary, by
one or more parameters. The keyword command parser which reads this file
will accept upper, lower or mixed case keywords. You may abbreviate
keywords to the shortest unambiguous string. \cs\ will tell you if the
abbreviation you use is ambiguous or a keyword is meaningless. It will also
tell you if the parameters you have entered with a keyword which requires
them are invalid.

When specifying filenames as parameters, you should be careful to use the
correct case if you are using an operating system such as UNIX which is
case-sensitive.

When specifying residues by number, note that the numbering in the PDB
file used for input is also used within \cs. However, residues will be
assigned to a chain (L, H, A, B, C, D, E, F) based on information from the
input sequence .PIR file. Thus, if you have a PDB file containing 2 chains 
each of 100 amino acids, numbered 1--200, \cs\ will
expect you to refer to these as residues L 1--100 in the first chain and
as H 101--200 in the second chain. Note that you will also be expected to
provide NTER and CTER residues in the PDB file, but it is conventional
in using \cs\ to give these numbers outside the range of the rest of the
structure. See Section~\ref{ex:coords} for an example.

The keywords which you may use are divided into two groups:
\begin{description}
\item[Global commands] These set general parameters. Effectively 
they replace the commands which were shared with CHARMm in CONGEN. In 
general, you will only need to use global commands to specify filenames and
to define the regions which are going to be rebuilt or which are to be
ignored in the constructions.
\item[Conformational search commands] These are used
to specify the exact residues to be constructed by conformational search,
the way in which the construction is to be performed and energy parameters
relating to the construction. 
\end{description}

\subsection{Global Commands}
Each of the global commands will be described in the following sections.
Some of these commands are {\em essential}; your \cs\ run will fail if they
are not specified. These are shown with the word {\bf Essential} as the 
first word in the description.

{\bf N.B.~The order of the global commands is important. If you do not
enter the commands in the order given, the \cs\ program is likely to exit
abnormally.} You should use the order:
\begin{itemize}
\item RESTOP
\item PARAMS
\item SIDETOP
\item GLYMAP
\item ALAMAP
\item PROMAP
\item PROCONS
\item PGP
\item SEQUENCE
\item COORDS
\end{itemize}

When specifying data files, unless you have them in your current directory,
you will have to specify a path by which to find them. The exact nature
of this path will depend on the operating system in use and the individual
installation of the software. For the local machine, \localmachine, at
\site, the data files are currently in the directory \localdat.

In the descriptions which follow, the pathname will be ommited from the
descriptions of files, you should insert the appropriate pathname for
your system.

\subsubsection{RESTOP {\em filename}}
\es
Specifies the residue topology file. This is used to store information
about the topology of standard residues and the atom types they contain.
You should select the standard file: {\tt topology}

\subsubsection{PARAMS {\em filename}}
\es
Specifies the CHARMm parameters file. This file contains parameters for
the interactions between atoms pairs (bonded and non-bonded), for
angles and both proper and improper torsion angles. You should select the
standard file: {\tt params}

\subsubsection{SIDETOP {\em filename}}
\es
Specifies the sidechain topology file. This contains topology information
for each of the sidechains. You should select the standard file:
{\tt sidetop}

\subsubsection{GLYMAP {\em filename}}
\es
Specifies the glycine Ramachandran energy map. You may select one of four
energy maps depending on the grid size you wish to use for the backbone
search. You should thus select one of:
\begin{itemize}
\item {\tt emapgly30}
\item {\tt emapgly15}
\item {\tt emapgly10}
\item {\tt emapgly5}
\end{itemize}

\subsubsection{ALAMAP {\em filename}}
\es
Specifies the alanine Ramachandran energy map. This is used as representative
of all amino acids aother than glycine and proline. You may select one of four
energy maps depending on the grid size you wish to use for the backbone
search. You should thus select one of:
\begin{itemize}
\item {\tt emapala30}
\item {\tt emapala15}
\item {\tt emapala10}
\item {\tt emapala5}
\end{itemize}

\subsubsection{PROMAP {\em filename}}
\es
Specifies the proline Ramachandran energy map. You may select one of four
energy maps depending on the grid size you wish to use for the backbone
search. You should thus select one of:
\begin{itemize}
\item {\tt emappro30}
\item {\tt emappro15}
\item {\tt emappro10}
\item {\tt emappro5}
\end{itemize}

\subsubsection{PROCONS {\em filename}}
\es
Specifies the proline constructor file. This contains data used in the
generation of the proline ring. You should select the standard file:
{\tt procns}

\subsubsection{PGP {\em filename}}
\es
Specifies a proton generation parameter file. This file is used to specify
where hydrogens need to be added to a structure and the geometry involved
in those additions. You should select the standard file: {\tt pgp}

\subsubsection{SEQUENCE {\em filename}}
\es
Specifies a sequence file containing sequence information for your protein.
This information is used to build the topology information for the protein.
The file is in standard PIR format. i.e. two title lines followed by the
sequence in one-letter code with each chain being terminated by an asterisk
({\tt *}). New lines and spaces in the sequence are ignored. The {\tt RESTOP}
and {\tt PARAMS} keywords must be specified before the {\tt SEQUENCE} keyword.
The chains are given labels by the software as they are read into the 
program. Because the system is designed primarilly for modelling antibodies,
the chains are labelled L, H, A, B, C, D, E, F. With antibodies, you should
thus specify the light chain then the heavy chain and any other chains
which are to be considered. \cs\ only supports up to
8 chains, but this is more than adequate for the majority of purposes.

\subsubsection{COORDS {\em filename}}
\label{sec:coords}
\es
Specifies the input PDB coordinate file for the protein. Regions of this
structure will be deleted by the {\tt CLEAR} command before being rebuilt by
conformational search. The {\tt PGP}, {\tt SEQUENCE} and {\tt RESTOP}
keywords must be specified before the {\tt COORDS} keyword.

The format of this file is standard PDB format with some minor additional
requirements:
\begin{enumerate}
\item Each chain must start with an {\tt NTER} residue containing atom types
{\tt HT1} and {\tt HT2}. These atoms should have `dummy' coordinates of
9999.000; the atom and residue numbers are unimportant, but the residue
number should be different from any other residue in the same chain.
\item The nitrogen atom of the first true residue of each chain should be 
renamed {\tt NT} rather than {\tt N}.
\item Each chain should end with a {\tt CTER} residue containing a single
{\tt OT2} atom with dummy coordinates. The atom and residue number is
unimportant, but the residue number should be different from any other
residue in the same chain.
\item The last true residue of each chain should have its carbonyl carbon
atom renamed from {\tt O} to {\tt OT1}.
\item The C-$\delta$ atoms in isoleucine residues must be named {\tt CD}
rather than the conventional {\tt CD1}.
\end{enumerate}
See Section~\ref{ex:coords} for an example.

\subsubsection{CLEAR {\em chain startres endres}}
Clears the coordinates for a range of residues. This is necessary before
rebuilding residues using the CGEN command. One may also choose to clear
other regions of the protein structure such that these regions do not
influence the building of the current region. For example, it is usual
practice to delete all the 6 CDRs from the combining site while any one
loop is being constructed.

\subsubsection{GLYEMAX {\em emax}}
Specifies the maximum allowed energy from the glycine Ramachandran map. 
\cs\ determines
the lowest energy in the Ramachandran map and adds this value to the lowest
energy. Only combinations of $\phi$/$\psi$ angles with energies lower than this
combined value are considered. The default is 100.0, but a more typical 
value is 2.00

\subsubsection{PROEMAX {\em emax}}
Specifies the maximum allowed energy from the proline Ramachandran map. 
\cs\ determines
the lowest energy in the Ramachandran map and adds this value to the lowest
energy. Only combinations of $\phi$/$\psi$ angles with energies lower than this
combined value are considered. The default is 100.0, but a more typical 
value is 2.00

\subsubsection{ALAEMAX {\em emax}}
Specifies the maximum allowed energy from the alanine Ramachandran map. 
\cs\ determines
the lowest energy in the Ramachandran map and adds this value to the lowest
energy. Only combinations of $\phi$/$\psi$ angles with energies lower than this
combined value are considered. The default is 100.0, but a more typical 
value is 2.00

\subsubsection{ERINGPRO {\em emax}}
Specifies the maximum allowed ring energy when constructing prolines.
The default is the largest single precision real number ($1.7\times10^{38}$),
but a more typical value is 50.00

\subsubsection{RESTART {\em restart}}
Specifies a restart string which should be enclosed in double inverted commas.
If a \cs\ run has been interrupted (for example, it has run out of disk
space or the computer has been switched off during the run) and a {\tt STATUS}
degree of freedom has been specified, the file written by the {\tt STATUS}
option
may be examined. The values specified in this file are given as a parameter
to the restart command (enclosed in double inverted commas) to cause the
conformational search to continue from the point at which the last status file
was written. It is then necessary to merge the conformations files which
are generated from the separate parts of the run.

\subsubsection{DEBUG {\em system level}}
Switches on debugging information. {\em system\/} may be one of the keywords:
{\tt PRIN}, {\tt ALLST}, {\tt ALLOC}, {\tt ALLHP}, {\tt CLSC}, or {\tt CGEN}.
Some of these options are now redundant and others are only of interest to
persons modifying \cs. The only options of any real interest are
{\tt CLSC}, and {\tt CGEN}. The {\em level\/} is a value bewteen 0 and 10;
higher values generate more copious volumes of output. Specifying a value of
10 for {\tt CGEN} will write to the disk continuously and will fill it as
quickly as possible!

\subsubsection{EPS {\em dielectric}}
Specifies the dielectric constant to be used in the evaluation of the 
electrostatic component of sidechain energies. (Default: 50.00)

\subsubsection{CUTNB {\em distance}}
Specifies the non-bonded cutoff distance in \ang. Non-bonded interactions
between atoms separated by more than this amount will not be considered.
(Default: 5.00)

\subsubsection{CUTHB {\em distance}}
Specifies the hydrogen bond cutoff distance in \ang. Hydrogen bond interactions
between atoms separated by more than this amount will not be considered.
(Default: 4.5)

\subsubsection{CUTHA {\em angle}}
Specifies the hydrogen bond cutoff angle in degrees. Hydrogen bond interactions
between atoms forming an angle of greater than this value will not be 
considered. (Default: 90.0)

\subsubsection{ECHO}
When a PDB file is read using the {\tt COORDS} keyword, certain modifications
such as filling in coordinates for the {\tt NTER} and {\tt CTER} residues 
of each chain
and the addition of hydrogens are normally made. This option causes the 
modified coordinates to be written to the file {\tt reference.pdb}.

\subsubsection{NOHADD}
When a PDB file is read using the {\tt COORDS} keyword, certain modifications
such as filling in coordinates for the {\tt NTER} and {\tt CTER} residues 
of each chain
and the addition of hydrogens are normally made. This option stops these
modifications from being made, so the input coordinate file must already
contain these data.






\subsection{Conformational Search Commands}
The conformational search process is represented by a set of `degrees
of freedom' (See Section~\ref{sec:understanding} for more information).
The term is used very loosely and does not have its conventional meaning
of spatial freedom. Rather it is used to describe levels in the conformational
search tree. Some levels may not be involved with the generation of new
conformations, but may, instead relate to the writing of coordinates, reading
of previously generated coordinates, or writing of status information.

Each degree of freedom is specified by one of the keywords described below
falling between the {\tt CGEN} and {\tt END} commands which bracket the
conformational search procedure.

Note that the degrees of freedom become nodes in the search tree in the
exact order in which they are specified in these commands. Clearly you cannot
construct sidechains on a stretch of residues before the backbone has been
built and you cannot, therefore, place a {\tt SIDECHAIN} command before
the {\tt FORWARD}, {\tt REVERSE} and {\tt CHAIN} commands used to build the
backbone. (Though you could, of course, first build sidechains on some other 
region of the structure which already has its backbone defined.) Similarly
if you specify the {\tt WRITE} command before you have specified any
construction commands, you will get no coordinates in your conformation file.

The commands are described in the order in which they would be used in
a typical construction.

None of the commands below is actually essential, but if a basic set of
commands is not given, either no conformation generation will occur, or the
coordinates will not be written to a file. Thus some of commands below
are marked as {\bf Essential}, even though their ommision will not 
cause \cs\ to exit abnormally.

Note that you will need to specify a {\tt CLEAR} command before using the
{\tt LOOPS} or backbone construction commands.

\subsubsection{CGEN}
\es
This command is used to indicate the start of the specification of 
conformational search degrees of freedom.

\subsubsection{LOOPS {\em filename}}
Specifies that a previously written set of conformations should be read. This
may either be a conformation file created by a previous run of \cs\ or a 
set of database conformations extracted using the \abm\ package.

\subsubsection{FORWARD {\em maxevdw chain startres lastres closeres closeatm}}
Specifies a range of residues to be constructed in the `forward' direction
(i.e. from the N-terminus towards the C-terminus). 
The region must first have been cleared using the {\tt CLEAR} keyword.
The {\em maxevdw\/} 
parameter specifies the maximum allowed backbone van der Waals interaction
energy (typically 20.0). 
The {\em chain}, {\em startres\/} and {\em lastres\/} parameters
specify the range of residues which are to be built, while the {\em closeres\/}
and {\em closeatm\/} parameters are used to determine which atom should be
used for distance checks when building the residues. This `closing' atom is
effectively where the chain is `aiming' for and should normally be the
C$\alpha$ atom of the last residue in the constructed region.

\subsubsection{REVERSE {\em maxevdw chain startres lastres closeres closeatm}}
Specifies a range of residues to be constructed in the `reverse' direction
(i.e. from the C-terminus towards the N-terminus). 
The region must first have been cleared using the {\tt CLEAR} keyword.
The {\em maxevdw\/} 
parameter specifies the maximum allowed backbone van der Waals interaction
energy (typically 20.0). 
The {\em chain}, {\em startres\/} and {\em lastres\/} parameters
specify the range of residues which are to be built, while the {\em closeres\/}
and {\em closeatm\/} parameters are used to determine which atom should be
used for distance checks when building the residues. This `closing' atom is
effectively where the chain is `aiming' for and should normally be the
N atom of the second residue in the constructed region.


\subsubsection{CHAIN {\em maxevdw chain startres}}
Specifies the start of a three-residue section which is to be built by the
chain closure algorithm of G\={o} and Scheraga\cite{go:closure}. 
The region must first have been cleared using the {\tt CLEAR} keyword.
{\tt FORWARD} and {\tt REVERSE}
commands must be given such that only these three residues are left to be
built. The {\em maxevdw\/} parameter specifies the maximum allowed backbone 
van der Waals interaction energy (typically 100.0) while the {\em chain} 
and {\em startres\/} parameters specify the beginning of the three residue 
section to be built.

\subsection{STATUS {\em filename}}
Specifies that a status file should be written containing information on
the progress through the conformational search tree. The file will only
contain information about degrees of freedom specified up to this point.
The information in this file is used in conjunction with the {\tt RESTART}
command to recover data after \cs\ has exited abnormally (for example after
a system crash or when the user's disk space has filled up). Generally this
command is given before the {\tt SIDECHAIN} command.

\subsubsection{SIDECHAIN {\em maxevdw grid chain startres lastres method}}
Specifies the range of sidechains to be built. The backbone for this region
must exist whether already present in the structure or built by conformational
search using the {\tt FORWARD}, {\tt REVERSE} and {\tt CHAIN} keywords. It
is {\em not\/} necessary to clear the region first; if the region {\em is\/}
cleared, the backbone must be reconstructed before sidechains may be built.
The {\em maxevdw\/}
parameter specifies the maximum allowed van der Waals interaction energy
(typically 20.0), the {\em grid\/} parameter may either be the word {\tt MIN} 
or a value in degrees which is the step size for conformational search around
sidechain torsion angles. Selecting {\tt MIN} causes only the lowest energy
torsion angles to be examined (generally this is the best option). 
The {\em chain}, {\em startres} and {\em lastres} parameters specify the 
range of sidechains to be built, while the {\em method} parameter specifies
the sidechain construction method to be used. This must be one of {\tt ITER},
{\tt ALL}, {\tt FIRST}, {\tt INDEPENDENT}, or {\tt COMBINATION}. The {\tt
ITER} method, which is generally the method of choice, is described in 
Section~\ref{sec:understanding}. For descriptions of the other methods,
you are referred to Bruccoleri and Karplus\cite{bruc:congen} or
Martin\cite{martin:thesis}.

\subsubsection{WRITE {\em filename}}
\es
Specifies that atom positions generated at the current depth in the 
conformational search tree should be written to the specified file.

\subsubsection{END}
\es
Specifies the end of the conformational search degrees of freedom.

%%%%%%%%%%%%%%%%%%%%%%%%%%%%%%%%%%%%%%%%%%%%%%%%%%%%%%%%%%%%%%%%%%%%%%%%%%%
\section{Examples}
\subsection{PDB File format}
\label{ex:coords}
The following example shows the form required for a PDB file to be
read by \cs. The example shows two chains of an antibody, but in order
to keep the example short, shows only the first and last two true
residues from each chain.

Note the addition of {\tt NTER} and {\tt CTER} residues and the modifications
to the first and last true residues in each chain. See 
Section~\ref{sec:coords} for details.

\begin{verbatim}
ATOM   9998  HT1 NTER    0    9999.0009999.0009999.000
ATOM   9998  HT2 NTER    0    9999.0009999.0009999.000
ATOM      1  NT  ASP     1      14.916  12.012   0.335  0.00  0.00
ATOM      2  CA  ASP     1      13.738  11.236  -0.105  0.00  0.00
ATOM      3  CB  ASP     1      12.455  11.967   0.314  0.00  0.00
ATOM      4  CG  ASP     1      12.186  11.734   1.803  0.00  0.00
ATOM      5  OD1 ASP     1      11.458  10.795   2.183  0.00  0.00
ATOM      6  OD2 ASP     1      12.746  12.526   2.598  0.00  0.00
ATOM      7  C   ASP     1      13.826  10.889  -1.584  0.00  0.00
ATOM      8  O   ASP     1      14.486  11.580  -2.389  0.00  0.00
ATOM      9  N   ILE     2      13.143   9.816  -1.924  0.00  0.00
ATOM     10  CA  ILE     2      13.114   9.316  -3.365  0.00  0.00
ATOM     11  CB  ILE     2      13.797   7.940  -3.263  0.00  0.00
ATOM     12  CG1 ILE     2      13.855   7.088  -4.541  0.00  0.00
ATOM     13  CG2 ILE     2      13.206   7.107  -2.077  0.00  0.00
ATOM     14  CD  ILE     2      15.021   6.021  -4.377  0.00  0.00
ATOM     15  C   ILE     2      11.668   9.438  -3.782  0.00  0.00
ATOM     16  O   ILE     2      10.785   8.897  -3.086  0.00  0.00
...
ATOM    887  N   ARG   114     -18.208   1.673 -20.442  0.00  0.00
ATOM    888  CA  ARG   114     -19.206   0.637 -20.838  0.00  0.00
ATOM    889  CB  ARG   114     -19.679  -0.148 -19.648  0.00  0.00
ATOM    890  CG  ARG   114     -19.976   0.639 -18.386  0.00  0.00
ATOM    891  CD  ARG   114     -21.157   0.102 -17.659  0.00  0.00
ATOM    892  NE  ARG   114     -22.291  -0.056 -18.553  0.00  0.00
ATOM    893  CZ  ARG   114     -23.115  -1.109 -18.540  0.00  0.00
ATOM    894  NH1 ARG   114     -22.987  -2.107 -17.676  0.00  0.00
ATOM    895  NH2 ARG   114     -24.100  -1.168 -19.452  0.00  0.00
ATOM    896  C   ARG   114     -20.315   1.331 -21.603  0.00  0.00
ATOM    897  O   ARG   114     -20.270   2.570 -21.788  0.00  0.00
ATOM    898  N   ALA   115     -21.278   0.541 -22.055  0.00  0.00
ATOM    899  CA  ALA   115     -22.411   1.141 -22.809  0.00  0.00
ATOM    900  CB  ALA   115     -23.042   0.181 -23.757  0.00  0.00
ATOM    901  C   ALA   115     -23.370   1.709 -21.753  0.00  0.00
ATOM    902  OT1 ALA   115     -23.582   1.117 -20.689  0.00  0.00
ATOM   6666  OT2 CTER  666    9999.0009999.0009999.000
ATOM   7777  HT1 NTER  777    9999.0009999.0009999.000
ATOM   7778  HT2 NTER  777    9999.0009999.0009999.000
ATOM    903  NT  GLN   116      -2.989 -19.567   8.673  0.00  0.00
ATOM    904  CA  GLN   116      -3.213 -18.168   8.472  0.00  0.00
ATOM    905  CB  GLN   116      -4.406 -17.712   9.381  0.00  0.00
ATOM    906  CG  GLN   116      -5.450 -16.887   8.651  0.00  0.00
ATOM    907  CD  GLN   116      -4.848 -15.846   7.740  0.00  0.00
ATOM    908  OE1 GLN   116      -4.317 -14.828   8.175  0.00  0.00
ATOM    909  NE2 GLN   116      -4.949 -16.107   6.439  0.00  0.00
ATOM    910  C   GLN   116      -1.861 -17.592   8.923  0.00  0.00
ATOM    911  O   GLN   116      -1.125 -18.277   9.659  0.00  0.00
ATOM    912  N   VAL   117      -1.603 -16.346   8.537  0.00  0.00
ATOM    913  CA  VAL   117      -0.425 -15.599   8.851  0.00  0.00
ATOM    914  CB  VAL   117       0.391 -15.576   7.547  0.00  0.00
ATOM    915  CG1 VAL   117       1.135 -14.301   7.225  0.00  0.00
ATOM    916  CG2 VAL   117       1.386 -16.674   7.758  0.00  0.00
ATOM    917  C   VAL   117      -1.191 -14.336   9.180  0.00  0.00
ATOM    918  O   VAL   117      -1.510 -13.561   8.280  0.00  0.00
...
ATOM   1805  N   SER   231      -8.439  12.161  14.381  0.00  0.00
ATOM   1806  CA  SER   231      -7.945  13.106  15.323  0.00  0.00
ATOM   1807  CB  SER   231      -6.497  13.495  15.127  0.00  0.00
ATOM   1808  OG  SER   231      -5.994  14.291  16.184  0.00  0.00
ATOM   1809  C   SER   231      -8.878  14.307  15.082  0.00  0.00
ATOM   1810  O   SER   231      -8.932  14.973  14.032  0.00  0.00
ATOM   1811  N   SER   232      -9.714  14.508  16.091  0.00  0.00
ATOM   1812  CA  SER   232     -10.663  15.579  16.104  0.00  0.00
ATOM   1813  CB  SER   232     -11.826  15.111  15.208  0.00  0.00
ATOM   1814  OG  SER   232     -12.998  15.935  15.141  0.00  0.00
ATOM   1815  C   SER   232     -11.129  16.014  17.502  0.00  0.00
ATOM   1816  OT1 SER   232     -11.245  15.324  18.507  0.00  0.00
ATOM   8888  OT2 CTER  888    9999.0009999.0009999.000
\end{verbatim}

\subsection{Rebuilding 5 Residues}
This example illustrates the technique for rebuilding a section of 
5 amino acids numbered 24--28 in the light chain. The first residue is 
built in the forward direction,
the last in the reverse direction, with the three in the middle
being built by the chain-closure method.

\begin{verbatim}
! Data files
RESTOP   /usr/people/AbM/ABM/DAT/topology
PARAMS   /usr/people/AbM/ABM/DAT/params
SIDETOP  /usr/people/AbM/ABM/DAT/sidetop
ALAMAP   /usr/people/AbM/ABM/DAT/emapala30
GLYMAP   /usr/people/AbM/ABM/DAT/emapgly30
PROMAP   /usr/people/AbM/ABM/DAT/emappro30
PROCONS  /usr/people/AbM/ABM/DAT/procns
PGP      /usr/people/AbM/ABM/DAT/pgp
!
SEQUENCE example1.pir   ! Sequence for this run
COORDS   example1.pdb   ! Input structure file for this run
! Parameters for this run
GLYEMAX  2.00
ALAEMAX  2.00
PROEMAX  2.00
ERINGPRO 50.00
EPS      50.0
CUTNB    5.0
CUTHB    4.5
CUTHA    90.0
!
ECHO                    ! Write coordinate file after input
CLEAR L   24   28       ! Clear the range for reconstruction
! Start the actual conformational search specification
CGEN
   FORWARD    20.0 L 24 24 28 CA
   REVERSE    20.0 L 28 28 25 N
   CHAIN     100.0 L 25
   SIDECHAIN  20.0 MIN L 24 28 ITER
   WRITE      L24_28.cg
END
\end{verbatim}

\subsection{Rebuilding Onto Previous Conformations}
This example illustrates the method used in \abm\ to read a set of
database conformations for the base of a loop and then to reconstruct
the mid-section of the loop by conformational search. This is done
in the same way as the `real space renormalisation' (RSR) 
technique described by Bruccoleri\cite{bruc:congen,bruc:hy5}.

In RSR, one builds the outside part of a long loop (maybe the first 2
residues and the last 2 residues) in one run of the program. A specifed 
number of the lowest energy conformations (maybe 10) are then read back
into a second run of the program and used as a base on which to build 
the next residues.

\cs\ does not evaluate the energy of the loops, so no energy is written
into the conformation file. It is thus not possible to read back only
the lowest energy conformations. Should you wish to attempt RSR, you
would need an external program capable of reading the conformations
file generated in the first run and writing a new conformations file
containing only the low energy conformations.

When used with database loops as is the case in \abm, one always wishes
to scan {\em all\/} the loop conformations.

The following example rebuilds the middle 5 residues (27--31) of
a loop spanning residues 24--33 in the light chain. The example assumes
that a file exists containing database conformations for the base 
regions of the loop (24--26 and 32--33).

\begin{verbatim}
! Data files
RESTOP   /usr/people/AbM/ABM/DAT/topology
PARAMS   /usr/people/AbM/ABM/DAT/params
SIDETOP  /usr/people/AbM/ABM/DAT/sidetop
ALAMAP   /usr/people/AbM/ABM/DAT/emapala30
GLYMAP   /usr/people/AbM/ABM/DAT/emapgly30
PROMAP   /usr/people/AbM/ABM/DAT/emappro30
PROCONS  /usr/people/AbM/ABM/DAT/procns
PGP      /usr/people/AbM/ABM/DAT/pgp
!
SEQUENCE example2.pir   ! Sequence for this run
COORDS   example2.pdb   ! Input structure file for this run
! Parameters for this run
GLYEMAX  2.00
ALAEMAX  2.00
PROEMAX  2.00
ERINGPRO 50.00
EPS      50.0
CUTNB    5.0
CUTHB    4.5
CUTHA    90.0
!
ECHO                    ! Write coordinate file after input
CLEAR L   24   33       ! Clear the range for reconstruction
! Start the actual conformational search specification
CGEN
   LOOPS      example2.cgin           ! Database loops
   FORWARD    20.0 L 27 27 31 CA
   REVERSE    20.0 L 31 31 28 N
   CHAIN     100.0 L 28
   SIDECHAIN  20.0 MIN L 24 33 ITER
   WRITE      L24_33.cg
END
\end{verbatim}






\subsection{Rebuilding Sidechains}
This example illustrates the rebuilding of a set of sidechains
without any backbone reconstruction. Sidechains for residues 33--39
in the heavy chain are rebuilt and the positions of residues 50--60
are not considered during the construction.

Note that we do not clear the coordinates for a region in which 
only the sidechains are being built.

\begin{verbatim}
! Data files
RESTOP   /usr/people/AbM/ABM/DAT/topology
PARAMS   /usr/people/AbM/ABM/DAT/params
SIDETOP  /usr/people/AbM/ABM/DAT/sidetop
ALAMAP   /usr/people/AbM/ABM/DAT/emapala30
GLYMAP   /usr/people/AbM/ABM/DAT/emapgly30
PROMAP   /usr/people/AbM/ABM/DAT/emappro30
PROCONS  /usr/people/AbM/ABM/DAT/procns
PGP      /usr/people/AbM/ABM/DAT/pgp
!
SEQUENCE example3.pir   ! Sequence for this run
COORDS   example3.pdb   ! Input structure file for this run
! Parameters for this run
GLYEMAX  2.00
ALAEMAX  2.00
PROEMAX  2.00
ERINGPRO 50.00
EPS      50.0
CUTNB    5.0
CUTHB    4.5
CUTHA    90.0
!
ECHO                    ! Write coordinate file after input
CLEAR H   50   60       ! Clear the region to be ignored
! Start the actual conformational search specification
CGEN
   SIDECHAIN  20.0 MIN H 33 39 ITER
   WRITE H33_39SC.cg
END
\end{verbatim}

\end{document}
