\documentclass{article}
\usepackage{a4}

\newcommand{\pf}{\mbox{\bfseries ProFit}}
\title{ProFit Version 3.3}
\author{Dr.\ Andrew C.R.\ Martin, Dr.\ Craig T.\ Porter\\ University College London}
\date{Document First Written: 25th July, 1996 (University College %
London)\\%
Updated while at University of Reading\\%
Last updated: 27th April, 2020}

\begin{document}
\maketitle

%%%%%%%%%%%%%%%%%%%%%%%%%%%%%%%%%%%%%%%%%%%%%%%%%%%%%%%%%%%%%%%%%%%%%%
\section{Introduction and Methodology}

\pf\ (pronounced Pro-Fit, not profit!) is designed to be the ultimate
program for performing least squares fits of two or more protein
structures. It performs a very simple and basic function, but allows
as much flexibility as possible in performing this procedure. Thus one
can specify subsets of atoms to be considered, specify zones to be
fitted by number, sequence, or by sequence alignment.

Early versions of \pf\ did not try to address the question of sorting out 
equivalent
atoms for you beyond doing a sequence alignment. There are other
programs such as SSAP and GAFIT which address that problem. You must
specify which residues and atoms you consider to be equivalent
although the program supports internal sequence alignment to set the
zones automatically.

As of \pf\ V2.0, iterative updating of fitting zones is now
supported. Thus you may give a sequence alignment or just a small fragment 
to initiate the
fitting process (a minimum of 3 amino acids). Fitting is performed on
this region and then all residue pairs within 3\AA\ are included in
the fitting zones and the fitting is repeated. This iterates until the
C$\alpha$ RMSd converges to within 0.01\AA. This is particularly
useful in conjunction with the initial zone specification based on
sequence alignment. Convergence typically takes 3--4 cycles.

\pf\ V2.0 also introduced multiple structure fitting. The first
structure file is used as a reference set for the first fitting stage
but the coordinates are averaged after each stage to derive a template
used for subsequent fitting. i.e.\ Given $N$ files to fit, file 2 is
fitted to file 1 and an averaged structure, $A$, is calculated, file 3
is then fitted to $A$ and a new average, $A'$ is calculated. This
continues until all $N$ structures have been fitted. The whole
procedure iterates until convergence (typically 3 or 4 cycles).

As of \pf\ V3.3, you can now specify the model number that you want to
read from a PDB file. This is done by appending \% and a model number
to the filename. For example, {\tt pdb2kqn.ent\%5}.

The program will output an RMS deviation and optionally the fitted
coordinates.  RMS deviations over alternate zones and atoms may also
be calculated without performing a new fit. Thus the zones for
calculating the RMS deviation can be different from those used for
fitting.

While optimised for proteins, non-protein structures may also be
fitted if they are stored in the standard Protein Databank (PDB) format.

\pf\ is written to be as easily portable between systems as possible
and uses a command-driven interface. 

\pf\ uses the McLachlan fitting algorithm, essentially a steepest
descents minimisation, as described in McLachlan, A.D. (1982)
\emph{Rapid Comparison of Protein Structures}, \emph{Acta Cryst.}\
{\bfseries A38}, 871--873. This part of the code is based on an
implementation by Dr.\ Mike Sutcliffe.


In summary, \pf\ has the following features:

\begin{enumerate}
\setlength{\parsep}{0pt}
\setlength{\parskip}{0pt}
\setlength{\itemsep}{0pt}
\item Portability between different operating systems
\item Ability to specify atom subsets
\item Ability to specify zones:
\begin{itemize}
        \setlength{\parsep}{0pt}
        \setlength{\parskip}{0pt}
        \setlength{\itemsep}{0pt}
        \item Numerically
        \item By sequence
        \item By auto sequence alignment
        \item By iterative updating and optimization
\end{itemize}
\item Output RMS deviation over:
\begin{itemize}
        \setlength{\parsep}{0pt}
        \setlength{\parskip}{0pt}
        \setlength{\itemsep}{0pt}
        \item Fitted region
        \item Any other region
        \item Any other atom set
\end{itemize}
\item Optionally output fitted coordinates in PDB format
\item Integrated help facility
\item Fitting zones derived from sequence alignment
\item Iterative updating of fitting zones
\item Multiple structure fitting
\end{enumerate}

%%%%%%%%%%%%%%%%%%%%%%%%%%%%%%%%%%%%%%%%%%%%%%%%%%%%%%%%%%%%%%%%%%%%%%
\section{Version numbering}

From V2.6, version numbering in \pf\ adopted the following scheme. A
version of \pf\ has a version number of the form V$a.b.c.d$ where $c$
and $d$ are optionally present.

$a$ is the major version number. A change in this
represents a significant rewrite of \pf\ and/or the addition of a
major new feature set.

$b$ is the minor version number. A change in $b$ represents a new
feature added to \pf . It may also indicate a fix to a major
show-stopper bug.

$c$ is a bug-fix number. It indicates that a bug in the \pf\ code has
been fixed compared with the previous release.

$d$ is also a bug-fix number, but indicates that bugs have been fixed
in the Bioplib libraries used by \pf\ and not in \pf\
itself.  Alternatively, it may indicate a distribution bug (i.e.\
missing files for a distribution or a simple documentation change).

In V2.5.$x$, $b$ was used to represent a new feature \emph{or} a bug
fix while $c$ was used to indicate a bug-fix in Bioplib ($d$ is now
used for that purpose).

In earlier releases, the scheme was used much more loosely with $b$
being used both for new features and bug fixes in \pf\ or in Bioplib.
Lettered versions (e.g.\ V1.7g) were sometimes used for bug-fixes and
sometimes for internal non-released versions. Lettered versions are
now used exclusively for internal non-released versions.


%%%%%%%%%%%%%%%%%%%%%%%%%%%%%%%%%%%%%%%%%%%%%%%%%%%%%%%%%%%%%%%%%%%%%%
\section{Starting the program}

The program is started from the command line by typing the command:
\begin{verbatim}
   profit
\end{verbatim}
Once the program is started, you may read in structures to be
fitted. Alternatively, the PDB files may be specified on the command
line:
\begin{verbatim}
   profit reference.pdb mobile.pdb
\end{verbatim}
By default, \pf\ does not read HETATM records from the PDB file. This
may be changed from the command line by using the -h flag:
\begin{verbatim}
   profit -h
   profit -h reference.pdb mobile.pdb
\end{verbatim}
Alternatively, once in the program you may give the \verb1HETATOMS1 
command before reading in the structures (see Section~\ref{sec:read}).

\pf\ can run a script file, a text file of \pf\ commands, from the command 
line using the -f flag: 
\begin{verbatim}
   profit -f myscriptfile.txt
   profit -f myscriptfile.txt -h reference.pdb mobile.pdb
\end{verbatim}
Once in the program you may use the command \verb|SCRIPT myscriptfile.txt|
to run a script file (see Section~\ref{sec:script}). 

If compiled with XMAS\footnote{XMAS is an XML-like file format
developed at Inpharmatica, Ltd.\ which is designed for leaf-heavy data
such as protein structure data} file support, the \verb1-x1 flag may
be used to specify that the files named on the command line are XMAS
files instead of PDB files. Note that the program currently will only
write PDB format files.
\begin{verbatim}
   profit -x reference.xmas mobile.xmas
\end{verbatim}

If compiled with GUNZIP support
then the program can read gzipped PDB files. This will only work on
unix-like platforms and assumes that the \verb|gunzip| program is in
your path. Note that the uncompressed files will remain in \verb|/tmp|
with a name like \verb|readpdb_12345| where 12345 is a process
number. You will need to delete these regularly!


Once in the program, you issue commands by typing at the
keyboard. These commands may always be abbreviated to the minimum
non-ambiguous string. The program is mostly case insensitive; you may
mix upper and lower case at will, though uppercase will be used
throughout this documentation. The only times that \pf\ will be case sensitive are when dealing with file names or lowercase chain identifiers (see Section~\ref{sec:zone}).

Support for the GNU Readline library\footnote{The GNU Readline library is 
available from http://directory.fsf.org/project/readline/} 
was introduced in \pf\ V3.0 allowing you to edit the command line and to 
recall previous commands. Again, this is a compile time option.

You exit from the program by typing \verb1QUIT1.


%%%%%%%%%%%%%%%%%%%%%%%%%%%%%%%%%%%%%%%%%%%%%%%%%%%%%%%%%%%%%%%%%%%%%%
\section{Reading Structures}

\label{sec:read} 
\pf\ reads files in PDB format. If compiled with XMAS support, then
XMAS files may also be read, but only PDB files may be written. It
uses the concept of a \emph{reference} structure and a \emph{mobile}
structure. The reference structure remains static in space and the
mobile structure is fitted onto it.  When the files are specified on
the command line, the reference structure is specified first, the
mobile structure second.

Once in the program, you may read the reference structure using the
\verb1REFERENCE1 command and the mobile structure using the
\verb1MOBILE1 command. Using these commands causes the equivalent
current structure to be deleted from the program's memory
first. However, any zone and atom specifications (see
Sections~\ref{sec:atoms} and~\ref{sec:zone}) are not deleted.  For
example, you can read p3hfl.pdb as a new reference structure using the
command:
\begin{verbatim}
   REFERENCE p3hfl.pdb
\end{verbatim}
and read p3hfm.pdb as a new mobile structure with:
\begin{verbatim}
   MOBILE p3hfm.pdb
\end{verbatim}

If compiled with XMAS support, then the XMAS format is specified by
placing the keyword \verb1XMAS1 after the commands \verb1REFERENCE1 or
\verb1MOBILE1: 
\begin{verbatim}
   REFERENCE XMAS p3hfl.pdb
   MOBILE XMAS p3hfm.pdb
\end{verbatim}

When you read a structure containing insertions, you will receive a
warning message to this effect. This dates from when the program was
unable to handle residue specifications containing insertion codes,
but is still useful to draw your attention to the fact that they are
present.

Note that atoms with coordinates of 9999.00, 9999.00, 9999.00 will
be ignored during all calculations allowing atoms with undefined 
coordinates to be handled. 

When fitting multiple structures (new in \pf\ V2.0), you use the
\verb1MULTI1 command to read in the structures. See
Section~\ref{sec:multi}. 

%%%%%%%%%%%%%%%%%%%%%%%%%%%%%%%%%%%%%%%%%%%%%%%%%%%%%%%%%%%%%%%%%%%%%%
\section{Getting Help}

To get help within the programs simply type \verb1HELP1 and you will
be presented with a list of commands which the help facility knows
about. The \verb1ProFit>1 prompt will also change to \verb1Help>1. You
may then type the name of a command to get help on that
command. Typing \verb1HELP1 once in the help facility will repeat the
list of available help topics. Like the main command interface, the
help facility will accept upper or lower case and you may abbreviate
commands. If your abbreviation is ambiguous (i.e.\ more than one
command starts with the letters you have specified), help will be
supplied on all the commands which match\footnote{The one exception to
this is if the letters you supply are an abbreviation of {\tt HELP},
when the list of help topics will be shown again.}.

If the help text is longer than 21 lines, you will see a prompt saying
\begin{verbatim}
   More...
\end{verbatim}
in which case you should hit the Return (or `Enter') key to get the
next page of help.

Once at the \verb1Help>1 prompt, you should simply hit the Return (or
`Enter') key to get back to the main \verb1ProFit>1 prompt.

If you know the topic on which you need help, you may type the name of
the command after the \verb1HELP1 keyword at the main \verb1ProFit>1
prompt. After the help message is printed, you will be returned
directly to the \verb1ProFit>1 prompt. For example, if you want help
on the \verb1ZONE1 command, you may type:
\begin{verbatim}
   HELP ZONE
\end{verbatim}

Allied to help, is the \verb1STATUS1 command. This tells you the
current status of the program: what structures are loaded, fitting
zones, atoms and the like.


%%%%%%%%%%%%%%%%%%%%%%%%%%%%%%%%%%%%%%%%%%%%%%%%%%%%%%%%%%%%%%%%%%%%%%
\section{Fitting Structures}

Having read in a reference and a mobile structure, you actually fit
them by giving the \verb1FIT1 command. When you do this, you will get
a message like:
\begin{verbatim}
   Fitting structures...
   RMS: 0.366
\end{verbatim}
However, this will only work if the two structures are of identical
composition i.e.\ if the sequences are the same and the same atoms are
present in both. If there are any mismatches, the first such mismatch
will be reported and the RMS deviation will not be calculated.  

Since you will frequently need to fit non-identical structures, you
may use the \verb1ZONE1 and \verb1ATOMS1 commands to specify which
residues should be considered equivalent and which atoms should be
considered in the calculation.

If you are using zone or atom specifications, the RMS deviations will
be displayed over the atoms and zones specified in those commands.

Normally the fitting procedure will not be completed if there are
any mismatched atoms or atoms missing from one of the two structures. 
The program issues an error message about atoms missing in the mobile 
structure which are found in the reference structure. The 
\verb1IGNOREMISSING1 command causes the program to issue a warning
instead of an error and the fitting proceeds ignoring the mismatched
atoms. The default behaviour is restored by using the
\verb1NOIGNOREMISSING1 command.

%%%%%%%%%%%%%%%%%%%%%%%%%%%%%%%%%%%%%%%%%%%%%%%%%%%%%%%%%%%%%%%%%%%%%%
\section{Specifying Atom Subsets}

\label{sec:atoms}
The \verb1ATOMS1 command is used to specify a subset of atoms to be 
used in the calculations. It has the syntax:
\begin{verbatim}
   ATOMS atm[,atm]...
\end{verbatim}
i.e.\ you specify the \verb1ATOMS1 keyword followed by one or more
atom names separated by commas. A \verb1*1 may be used to specify all
atoms and a \verb1~1 or \verb1^1 may be placed at the beginning of the
specification to inverse the selection. For example, to fit only
C$\alpha$ atoms:
\begin{verbatim}
   ATOMS CA
\end{verbatim}
to fit N, C$\alpha$, C and O atoms:
\begin{verbatim}
   ATOMS N,CA,C,O
\end{verbatim}
to fit sidechains only (i.e.\ everything except N, C$\alpha$, C and O atoms):
\begin{verbatim}
   ATOMS ^N,CA,C,O
\end{verbatim}
to return to fitting all atoms:
\begin{verbatim}
   ATOMS *
\end{verbatim}

The PDB atom name field is 4 characters wide followed by a space.  The
first two characters are the right-justified element type, so for
normal protein and DNA atoms consist of a space followed by a N, C, O,
S or P. Thus the atom name field for a C$\alpha$ contains
`~CA~'. HETATMs such as calcium will contain the two characters CA in
the first two fields. i.e.\ `CA~~'. When you specify an atom type it is
matched against the atom name field from the \emph{second character onwards},
unless you preceed it with a \verb1<1. Thus to match a C$\alpha$ you use
\verb1CA1, but to match Calcium, you use \verb1<CA1. For example, as
stated above, to match C$\alpha$ atoms:
\begin{verbatim}
   ATOMS CA
\end{verbatim}
while to match calcium atoms
\begin{verbatim}
   ATOMS <CA
\end{verbatim}
and to match both C$\alpha$ and calcium:
\begin{verbatim}
   ATOMS <CA,CA
\end{verbatim}


Wildcards are also allowed. A \verb1%1 or a \verb1?1 may be used to
match a single letter at any point in the specification while a
\verb1*1 may be used to match all remaining characters (thus \verb1C*1
is allowed, but \verb1*G1 is not). These special characters may be
escaped by preceding them with a \verb1\1. For example to fit all
carbons:
\begin{verbatim}
   ATOMS C*
\end{verbatim}
or to match all atoms at the $\gamma$ position:
\begin{verbatim}
   ATOMS ?G*
\end{verbatim}
and to match the \verb1C4*1 atoms in DNA:
\begin{verbatim}
   ATOMS C4\*
\end{verbatim}

If atom names contain spaces (e.g.\ in heme groups) the whole atom
specification must be enclosed in double inverted commas:
\begin{verbatim}
   ATOMS "N A,N B,N C"
\end{verbatim}


%%%%%%%%%%%%%%%%%%%%%%%%%%%%%%%%%%%%%%%%%%%%%%%%%%%%%%%%%%%%%%%%%%%%%%
\section{Specifying Zones}

\label{sec:zone}
The \verb1ZONE1 command is used to specify zones in the two structures
which are considered equivalent. The complete syntax for the command
is:

\begin{verbatim}
   ZONE CLEAR|((*|(X...[,n][/m])|(j-k))[:(*|(X...[,n][/m])|(j-k))])
\end{verbatim}
where \verb1X...1 is an amino acid sequence, \verb1n1 is a number of
residues, \verb1m1 is the occurrence number, \verb1j1 and \verb1k1 are
residue specifications of the form \emph{[chain][.]resnum[insert]}. Items
in square brackets are optional and alternatives are marked by a
\verb1|1 and grouped in parentheses.

\verb1ZONE1 commands are cumulative. Thus each zone you specify is
added to those currently active. To clear all zones (i.e.\ fit all
residues), the \verb1ZONE CLEAR1 or \verb1ZONE *1 command may be
given. To clear a single zone, the DELZONE command can be used 
(see the end of this section).

When a new zone is added, a warning message is displayed if the new zone 
overlaps an existing zone. Overlapping zones will be flagged with \verb1*1 
when using the \verb1STATUS1 command.

Although it appears complex, the syntax is actually very simple and
consists of two identical sections separated by a colon (:). The left
half is applied to the reference structure and the right half to the
mobile structure. In its simplest form, the right hand half of the
expression is absent and the specification is applied to both
reference and mobile structures. For example:
\begin{verbatim}
   ZONE 24-34
\end{verbatim}
will set the zone to include residues 24--34 in both structures. If you
wanted to fit 24--34 in the reference structure with 25--35 in the
mobile structure, this simply becomes:
\begin{verbatim}
   ZONE 24-34:25-35
\end{verbatim}
Single residues can be specified using the same syntax:
\begin{verbatim}
   ZONE 44-44:55-55
\end{verbatim}

You may also specify chain names and insertion codes. The chain name is placed
before the residue number and the insertion code afterwards. For example:
\begin{verbatim}
   ZONE L25A-L30
\end{verbatim}
fits residues 25A--30 in the L chain of both structures. Optionally,
the chain name may be separated from the residue number using a full
stop. For example:
\begin{verbatim}
   ZONE L.25A-L.30
\end{verbatim}
Using the full stop also makes the statement case-sensitive. In practice, the
full stop separator is used with numeric chain names to separate the
chain name from the residue number and with lowercase chain names.
\begin{verbatim}
   ZONE 1.25-1.30
   ZONE b.1-b.60:A.1-A.60
\end{verbatim}

Simple wildcards may also be used. For example
\begin{verbatim}
   ZONE H*:B*
\end{verbatim}
fits the reference H chain with the mobile B chain,
\begin{verbatim}
   ZONE -10:50-59
\end{verbatim}
fits from the first residue to residue 10 in the reference structure
with 50--59 in the mobile structure.
\begin{verbatim}
   ZONE *:1-100
\end{verbatim}
fits all residues in the reference structure with 1-100 in the mobile
structure.  

If the structure file contains negatively numbered residues and you
are using residue numbering, you can escape the minus sign in the
residue number using a backslash:
\begin{verbatim}
   ZONE \-4-10:\-1-13
\end{verbatim}
will fit residues $-4$ to 10 in one structure with $-1$ to 13 in the
other.

Alternatively, you may specify the zones to be fitted by giving a
sequence fragment. Together with that fragment, you may specify the
number of residues to consider starting at that point. If the fragment
occurs more than once in the sequence you may specify which occurrence
you wish to consider. For example:
\begin{verbatim}
   ZONE CAR:VNS
\end{verbatim}
fits the first occurrence of CAR in the reference set with first
occurrence of VNS in the mobile set;
\begin{verbatim}
   ZONE CAR,10:VNS,10
\end{verbatim}
fits 10 residues starting at the  first occurrence of CAR in the 
reference set with 10 residues from the first occurrence of VNS in 
the mobile set;
\begin{verbatim}
   ZONE CAR,5/2
\end{verbatim}
fits 5 residues from second occurrence of CAR in both structures;
\begin{verbatim}
   ZONE 24-34:EIR,ll
\end{verbatim}
fits 24-34 in the reference set with 11 residues starting at the 
first occurrence of EIR in the mobile set.

By default, \pf\ works in `Residue Number' mode, i.e.\ the numbers
used in zone commands are the numbers seen in the PDB file. The
alternative mode is `Sequential' mode where residues are numbered
sequentially throughout the structure (including throughout multiple
chains). Any chain names appearing in zone specifications will be
ignored in Sequential mode. To switch mode, you use the {\tt NUMBER
SEQUENTIAL} or {\tt NUMBER RESIDUE} commands.

The \verb1DELZONE1 command specifies zones to be deleted from the user-defined 
list of fit zones. \verb1DELZONE1 uses the same syntax as the \verb1ZONE1 
command. The command matches the specified zone with a zone in the user-defined
list of fitting zones and deletes the matching zone from the list. Entering 
either \verb1DELZONE ALL1 or \verb1DELZONE *1 will delete all user-defined 
zones.

\subsection{Sequence Alignment}
%------------------------------
Another way of specifying zones is to let the program do it. \pf\ 
allows you to perform a simple Needleman and
Wunsch sequence alignment and to apply zones automatically derived
from that sequence alignment. This is done by issuing the \verb1ALIGN1
command. The sequence alignment is displayed, any currently active
fitting zones are cleared and replaced by zones derived from the
alignment. Additional zones may also be specified in the usual way.

As of Version 3.0, \pf\ offers a choice of three alignment options:

\begin{enumerate}
\setlength{\parsep}{0pt}
\setlength{\parskip}{0pt}
\setlength{\itemsep}{0pt}

\item  The default alignment option is a chain-by-chain alignment where the 
first chain in the mobile is aligned with first chain in the reference, the 
second chain in the mobile is aligned with the second chain in the reference, 
and so on. If the number of chains does not match then a warning is issued.

\item The \verb1ALIGN1 \verb1WHOLE1 command gives a whole sequence alignment. 
The whole sequence (regardless of chain ID) is aligned. If the fitting zones 
assigned in this manner extend over more than one chain the zones are split 
into smaller zones at the breaks between chains.
This may be useful if a sequence has been split into fragments.

\item If a zone definition is supplied to the \verb1ALIGN1 command then \pf\ 
will perform an alignment over the defined region to assign fitting zones. 
(See Section~\ref{sec:zone} for the syntax for defining zones.)
 
It is also possible to append new zones onto the end of the zone list (rather 
than overwriting the current zone list) by adding \verb1APPEND1 after the zone 
definition.

For example one could use following commands: 
\begin{verbatim}
   ALIGN A*:B*
   ALIGN B*:A* APPEND
\end{verbatim}
to align chain A with chain B and then B with A. 
This is useful when chains appear in different orders in the PDB files.

When doing multiple fitting, it is not possible use the colon notation
to define regions on both the reference and mobile structures. This is 
the same restriction as the \verb1ZONE1 command 
(see Section~\ref{subsec:multi_zone}).
\end{enumerate}

Clearly, it will normally be necessary to use the \verb1ATOMS1 command
to specify that only backbone or C$\alpha$ atoms are included in the
fitting. The \verb1TRIMZONES1 command can also be used when doing multiple 
structure fitting to ensure that the fitting zones are identical for all 
mobile structures. (See Section~\ref{subsec:multi_zone})

The \verb1GAPPEN1 command allows you to specify an integer gap penalty
and gap extension penalty for the sequence alignment performed by the 
\verb1ALIGN1 command. The default values for the gap penalty and gap 
extension penalty are 10 and 2 respectively.

\subsection{Reading an Alignment}
%--------------------------------

If you have an alignment performed outside \pf\ you may use this to
specify the equivalent zones. Any previously defined fitting zones are
automatically cleared first. As of \pf\ V3.0, the \verb1READALIGN1 command 
can be used with structures having more than one chain.  

The alignment should be a file in PIR format using -- characters to
align the sequences. The two sequences are represented by separate
entries, i.e.\ each must have a header of the form:
\begin{verbatim}
   >P1;xxxxxx
   title text .......
\end{verbatim}

When reading an alignment file for aligning a reference structure with a 
single mobile structure, the first sequence will be assumed to be that of 
the reference structure and the second is that of the mobile structure. Any
other sequences in the file are ignored. Chain Breaks in a sequence are 
indicated with a \verb1*1.
\begin{verbatim}
   >P1;REFSEQ
   Reference Sequence - first.pdb
   WILLIAM*H-ARTNELL-*

   >P1;M_0001
   Mobile Sequence - second.pdb
   --PATRI-K*TR--GHTN*
\end{verbatim}

The \verb1READALIGNMENT1 command is also used to read in the PIR files 
containing a multiple sequence alignment.
When performing a multiple structure fit, the first sequence
\emph{must appear twice} in the sequence alignment file. This is
because it is used as both the initial reference and first mobile set:
\begin{verbatim}
   >P1;REFSEQ
   Reference Sequence - first.pdb
   ----WILLIAM*H-ARTNELL-*

   >P1;M_0001
   Mobile Sequence - first.pdb
   ----WILLIAM*H-ARTNELL-*

   >P1;M_0002
   Mobile Sequence - second.pdb
   ------PATRI-K*TR--GHTN*

   >P1;M_0003
   Mobile Sequence - third.pdb
   PERTWEE---------------*
\end{verbatim}

Note that a bug in using the  \verb1READALIGNMENT1 with multiple
structure fitting was fixed in V2.3. (The bug caused the program
to crash if a deletion appeared in the same place in two or
more of the sequences.)

\subsection{Limiting Zones Read From an Alignment}
%-------------------------------------------------

When obtaining fit zones from a sequence alignment, either from
\verb1ALIGN1 or from \verb1READALIGNMENT1, it can be useful to limit
the zones of residues used. Normally all aligned residue pairs will be
used.

For example, if the alignment were:
\begin{verbatim}
                       1         2         3
              123456789012345678901234567890123
              ASAHSTGEHNM--PLELLGHISLAM---NPRTY
              ---HSTADHNLRTPLEVLG--SLAMEDRQPRTY
\end{verbatim}
the zones would normally be taken from the following positions
in the alignment: 4-11, 14-19, 22-25, 29-33

By using the command:
\begin{verbatim}
      LIMIT 20 28
\end{verbatim}
only the zone from 22-25 would be included.

This is particularly useful in conjunction with the ITERATE command
(Section~\ref{subsec:iterate}) and when fitting multiple structures
(Section~\ref{sec:multi}).

The \verb|LIMIT OFF| command restores the default behaviour of
deriving the zones from the whole alignment.




\subsection{Iterative Updating of the Fitting Zones}
%---------------------------------------------------

\label{subsec:iterate}

The \verb1ITERATE1 command switches on the iterative updating of
fitted zones during subsequent \verb1FIT1 commands. The \verb1ITERATE1
command may be followed by an optional parameter to specify the cutoff
used to include or exclude pairs from the zones.  (\verb1ITERATE OFF1
is used to switch it off again.)

Note that this immediately does an \verb1ATOMS CA1 since iteration of
zones is only performed on C$\alpha$ atoms. The program gives an
informational message to this effect. See notes below if you want to
calculate an RMSd over other atoms.

After the initial fit on the specified zones, the zones are updated
such that residue pairs with C$\alpha$ atoms within a specified cutoff
(default 3.0\AA) are included and those more distant are excluded. The
optimum set of equivalences is obtained using a dynamic programming
method.

After updating the zones, the structures are refitted and the
procedure iterates to convergence of $<0.01$\AA, (typically 3 or 4
cycles). The RMSd on C$\alpha$ atoms is shown after each cycle unless
the \verb1QUIET1 command is given before running \verb1ITERATE1.

You may specify a minimal initial zone of say 3 amino acids on which
to fit first. The zone iteration will expand the zones until as many
residues as possible can be equivalenced. Alternatively, this option
is particularly useful in conjunction with the \verb1ALIGN1
command. Using \verb1ALIGN1 followed by \verb1ITERATE1 gives a
particularly convenient method of fitting two arbitrary structures.

As stated above, the \verb1ITERATE1 command implies 
\verb1ATOMS CA1. Having fitted on C$\alpha$ atoms, you can of course
display the RMSd over other atom sets in the usual way using the
\verb1RATOMS1 command (e.g.\ \verb1RATOMS N,CA,C,O1 will display the
backbone RMSd).

Should you wish to refit on another atom set using the iterated zones,
simply use \verb1ITERATE OFF1 to switch off iteration, select the atom
set required using the \verb1ATOMS1 command and use \verb1FIT1 to
refit the structures in the usual way. For example, to fit on backbone
atoms:

\begin{verbatim}
   ITERATE OFF
   ATOMS N,CA,C,O
   FIT
\end{verbatim}


\subsection{Fitting Zones based on the Temperature Factor Column.}
%-----------------------------------------------------------------

\label{sec:bzone} 
Note that this use of the B-value column is not compatible with the commands described in Section~\ref{sec:modfit}.

It is possible to define zones by flagging residues in the temperature 
factor column of the PDB file using the \verb1BZONE1 command. Zones are marked 
using a positive whole numbers while zeros are ignored. Multiple zones can be 
marked using additional numbers. 
So, residues with the B-factor set to 1 will be fitted with one another, 
residues with the B-factor set to 2 will be fitted with one another, etc.

Assignment of zones is carried out in two ways: 

If only the reference 
structure is marked then the same set of residue numbers will be added as a 
fitting zone in both the reference and mobile structure. 

If both the reference and the mobile 
structure are marked then fitting zones are assigned by scanning through and 
setting zones for corresponding continuous stretches of flagged residues in 
either the reference or mobile structures. 


\subsection{Centre of Fitting}
%-----------------------------

\label{sec:setcentre}
The default method for fitting is to centre the fit around the centre of 
geometry of the fit atoms. Alternatively, fitting can be centred around 
the centre of geometry of a residue
specified by the \verb1SETCENTRE1 (or \verb1SETCENTER1) command.


\begin{verbatim}
   SETCENTRE CLEAR|(*|i[:j])
\end{verbatim}
where i and j are residue specifications of the form \emph{[chain][.]resnum[insert]}.
Items in square brackets are optional and alternatives are marked by a
\verb1|1 and grouped in parentheses.

The command:
\begin{verbatim}
   SETCENTRE 24:35
\end{verbatim}
will centre the fit around residue 24 of the reference structure and residue 25 of the mobile structure. The mobile residue number can be omitted. For example:
\begin{verbatim}
   SETCENTRE 33
\end{verbatim}
will centre the fit around residue 33 of the reference structure and residue 33 of the mobile structure.

Entering \verb1SETCENTRE CLEAR1 or\verb1 SETCENTRE *1 will clear the centre 
residue.

\subsection{Distance Cutoff for RMSd Calculations}
%-------------------------------------------------

\label{sec:distcutoff}
The \verb1DISTCUTOFF1 command specifies a distance cutoff for ignoring atom 
pairs outside a specified distance when calculating RMSd.

\begin{verbatim}
   DISTCUTOFF [cutoff|ON|OFF]
\end{verbatim}

The \verb1DISTCUTOFF1 command specifies a distance cutoff for ignoring atom 
pairs outside a specified distance when calculating RMSd. Entering 
\verb1DISTCUTOFF ON1 or \verb1DISTCUTOFF OFF1 will turn the distance cutoff on 
or off. Entering \verb1DISTCUTOFF 2.51 will set the value of the distance 
cutoff to 2.5 Angstroms and turn the distance cutoff on. A warning is displayed
if the distance cutoff is set to zero and turned on. Note that the cutoff is 
only applied to the final calculation of RMSD and not to the fitting.




%%%%%%%%%%%%%%%%%%%%%%%%%%%%%%%%%%%%%%%%%%%%%%%%%%%%%%%%%%%%%%%%%%%%%%
\section{Multiple Structure Fitting}

\label{sec:multi}

The \verb1MULTI1 command allows a multiple set of structures to be
read in for fitting. The filename specified for \verb1MULTI1 is a
`file of files' i.e.\ it contains a list of filenames which will be
read.

As with single structure fitting, a model number may be appended after
a \% (e.g. {\tt pdb2kqn.ent\%2}).

\verb1MULTI1 is used in place of \verb1REFERENCE1 and \verb1MOBILE1 to
read in a set of structure files. The first structure file is used as
a reference set for the first fitting stage, but the coordinates are
averaged after each fitting stage to derive an averaged template used
for subsequent fitting.

i.e.\ Given $N$ files to fit, file 2 is fitted to file 1 and an
averaged structure, $A$, is calculated, file 3 is then fitted to $A$
and a new average, $A'$ is calculated. This continues until all $N$
structures have been fitted. The whole procedure iterates until
convergence (typically 3 or 4 cycles).

\pf\ V3.0 changes the default method of calculating the average template. 
As each new mobile structure is added, the degree of change in the averaged 
structure is inversely proportional to the total number of mobile structures. 
Consequently, outlying structures should have less effect on the averaged 
reference structure.

Normally, the coordinates of the first structure in the \verb1MULTI1 list 
are taken as the starting point for the averaged reference structure. It is 
possible however, to select another mobile structure as the initial reference 
structure using the \verb1SETREF1 command. 
For example, \verb1SETREF 31 will use the third mobile structure as the 
reference strcture.
If no structure number is 
specified, then the \verb1SETREF1 command carries out an all vs. all 
comparison and the coordinates 
of the mobile structure with the least overall RMSD to all the other 
mobile structures are selected as the initial reference structure.

Multiple structures can be fitted with either the \verb1FIT1 or 
\verb1ORDERFIT1 command. The \verb1ORDERFIT1 command (new in V3.0) will 
perform the multiple structure fit in a similar manner to the \verb1FIT1 
command but fitting the most similar structures first. As the averaged 
template is updated with each new structure fitted, the order of fitting 
has a (small) influence on the template. The ORDERFIT command (possibly 
along with the SETREF command) can provide a standardized fitting scheme.


Progress and RMSds are reported at each iteration unless the \verb1QUIET1
command is used.

By default, RMSDs, pairwise distances and transformation matrices are given in 
relation to the first mobile structure. The \verb1MULTREF1 command will set 
\pf\ to give results in relation to the averaged reference structure rather 
than the  first mobile structure (\verb1MULTREF1 \verb1OFF1 restores the 
default behaviour).

The resulting fitted files are written with the \verb1MWRITE1 command.
Note that there is no ``reference'' set in the sense used for normal
2-structure fitting; fitted versions of all $N$ files will be written
since the reference set is actually an averaged template used purely as a 
guide for fitting. 

The averaged template can be written to a file using the \verb1WRITE1 
\verb1REF1 command. As it is a simple numerical average of the cartesian 
coordinates however, taking the reference structure generated by ProFit as a 
representation of an actual geometry/conformation accessible by the structure 
should be done with caution.

When the \verb1MWRITE1 command is used, the output filenames are the
same as the input files, but with the extension replaced by that
specified in the \verb1MWRITE1 command. If no extension is specified,
then `.fit' will be used. If the input structure files contained no
extension, then the extension specified will be appended to the
filenames.

If you have specified a model number for the input files, then this
will be appended to the output filename except for model 1.

Note that since only the extension is changed when writing back the
fitted files, you must have permission to write to the directory from
which the original files were read.

Multiple-structure fitting is particularly effective in combination
with the \verb1ITERATE1 command (see Section~\ref{subsec:iterate})
which refines the fitting zones iteratively. This can lead to
extremely good multiple structures fits.

Note that multiple structure fitting and zone iteration can be very
slow as these have been added to the earlier pair-wise fitting
engine. An increase in speed needs a complete re-design of the code.


\subsection{Specifying Zones With Multiple Structure Fitting}
%------------------------------------------------------------

\label{subsec:multi_zone}

Currently, the ZONE command may only be used with multiple structure
fitting when the same zone specification may be applied to every
structure. i.e.\ You cannot specify a zone for each structure
separating the zones with a colon (:)

Thus, the following are legal zones:
\begin{verbatim}
  ZONE 20-30
  ZONE C,3
\end{verbatim}
while the following are not:
\begin{verbatim}
  ZONE 24-34:25-35
  ZONE CAR:VNS
  ZONE 24-34:EIR,11
\end{verbatim}

For normal use, it is recommended that the \verb1ALIGN1, \verb1TRIMZONES1 and
\verb1READALIGNMENT1 commands (possibly in conjunction with the
\verb1LIMIT1 command) are used for specifying zones when fitting
multiple structures.


As of \pf\ V3.0, the \verb1TRIMZONES1 command can be used in conjuction with 
the \verb1ALIGN1 command. The \verb1ALIGN1 command performs a pairwise 
alignment for each of the mobile structures with the reference. 
Although fitting each mobile 
using an individualized set of zones offers the best fitting for each mobile 
to the reference, there may be times when a like vs. like comparison is 
required. If the number of residues used for fitting varies, RMS 
deviation cannot be directly compared between structures.

To allow for a like vs. like comparison, the \verb1TRIMZONES1 command resets 
the fitting zones for each mobile structure to include only fitting residues 
that are common to all the mobile structures. Thus, by ensuring that the 
fitting zones are the same for each mobile, the \verb1TRIMZONES1 command 
allows for a like vs. like comparison. 

When using \verb1READALIGNMENT1 with
multiple structures, the first sequence \emph{must appear twice} in
the alignment file. This is because it is used as both the first
reference and mobile set.

Note that a bug in using the  \verb1READALIGNMENT1 with multiple
structure fitting was fixed in V2.3. (The bug caused the program
to crash if a deletion appeared in the same place in two or
more of the sequences.)

\subsection{All Versus All Comparisons}
%--------------------------------------

\label{subsec:multi_allvall}

As of \pf\ V3.0, it is possible to perform an all versus all comparison of 
the mobile structures when fitting multiple structures. The \verb1ALLVSALL1 
command requires that the fitting zones set are identical for all mobile 
stuctures and automatically resets the fitting zones using the 
\verb1TRIMZONES1 command.

Results are presented as tab-delimited text suitable for loading into a 
spreadsheet. If the optional filename parameter is given, output is 
directed to the specified file. If a filename is not specified, or the file 
cannot be opened, output appears on the screen. If the filename begins with 
a pipe character ($\mid$), the results are piped into the specified program.
This is particularly useful with the \verb1more1 (or \verb1less1) Unix
command. 


%%%%%%%%%%%%%%%%%%%%%%%%%%%%%%%%%%%%%%%%%%%%%%%%%%%%%%%%%%%%%%%%%%%%%%
\section{Calculating the RMSd Over Other Zones and Atoms}
Having fitted the structures using the \verb1ZONE1 and \verb1ATOMS1
commands to specify which residues and atoms should be included in the
fitting, the RMS deviation may then be calculated over a different
region of the structure and/or a different atom set.

This is achieved using the \verb1RZONE1 and \verb1RATOMS1
commands. The syntax of these commands is identical to that of the
\verb1ZONE1 and \verb1ATOMS1 commands described in
Sections~\ref{sec:atoms} and~\ref{sec:zone}.

As each \verb1RZONE1 or \verb1RATOMS1 command is given, the RMS
deviation is reported over the new set of zones or over the new atom
set. Don't forget the \verb1RZONE1 commands are cummulative, like the
\verb1ZONE1 commands. Note that the \verb1RZONE *1 or \verb1RZONE1
\verb1CLEAR1 behaves slightly differently from \verb1ZONE *1 or
\verb1ZONE1 \verb1CLEAR1 since it resets the zones to be the same as
those specified for the fitting using \verb1ZONE1, \verb1ALIGN1 or
\verb1READALIGNMENT1 commands.

The \verb1DELRZONE1 command specifies zones to be deleted from the list of 
user-defined zones for calculating the RMSd. The \verb1DELRZONE1 command uses 
the same syntax as the \verb1DELZONE1 command. The command matches the 
specified zone with a zone in the user-defined list of RMSd calculation zones 
and deletes the matching zone from the list. Unlike the \verb1RZONE1 command, 
entering either \verb1DELRZONE ALL1 or \verb1DELRZONE *1 will delete all 
user-defined RMSd calculation zones rather than returning to the default 
condition where the RMSd calculation zones are set to the user-defined fitting 
zones. Thus avoiding the somewhat counterintuitive situation where deleting 
the last RMSd zone restores all RMSd zones. If no RMSd calculation zones are 
defined then \pf\ will calculate the RMSd over all residues.

%%%%%%%%%%%%%%%%%%%%%%%%%%%%%%%%%%%%%%%%%%%%%%%%%%%%%%%%%%%%%%%%%%%%%%
\section{Obtaining Output}

\subsection{The Fitted Structure}
%--------------------------------
The fitted mobile structure may be written to a file in PDB format
using the \verb1WRITE1 command:
\begin{verbatim}
   WRITE fitted.pdb
\end{verbatim}
If the first character of the filename is a pipe character ($\mid$),
then the results will be piped into the specified program. For
example:
\begin{verbatim}
   WRITE |less
\end{verbatim}
will cause the coordinates to be displayed on the screen using the
\verb1less1 pager program.

The reference set may also be written:
\begin{verbatim}
   WRITE REFERENCE ref_fitted.pdb
\end{verbatim}
(only the three letters `REF' of the REFERENCE parameter are
required). This is only useful if the CENTRE command has been used
(see below).

\subsection{Centering the Coordinates}
%-------------------------------------
By default, the mobile structure is moved to the coordinate frame of
the reference set. If the \verb1CENTRE1 (or \verb1CENTER1) command is
given then the centre of geometry of the fitted coordinates will be
located at the origin.

If a residue has been set as the centre of fitting using \verb1SETCENTRE1 
(see Section~\ref{sec:setcentre}) then that residue will be moved to the 
origin when the \verb1CENTRE1 command is used.

If only two structures are fitted then the \verb1WRITE REFERENCE1
command must be used to write the reference set in the origin-centred
coordinate frame. If multiple structures are fitted and written using
\verb1MWRITE1 then the reference set will be written automatically.





\subsection{Details of the Fitting}
%----------------------------------

More details about the fitting may be obtained by using the
\verb1MATRIX1 command. This displays the centres of geometry, the
rotation matrix and the translation vector which is the vector between
the centres of geometry. Thus to superimpose the mobile structure onto
the reference structure using these data, you should translate the
mobile set to the origin, apply the rotation matrix, translate back to
the original centre of geometry and finally apply the translation
vector.

Note that the rotation matrix is not orthogonal and cannot therefore
be used to extract Euler angles. This is a result of the fitting
method used.

The \verb1NFITTED1 command displays the number of atom pairs which
were fitted in the last fitting operation. Note that this will not
be the number of residues fitted unless you are only fitting one
atom type per residue (typically C$\alpha$ atoms).


\subsection{By-residue RMS Deviation}
%------------------------------------

The \verb1RESIDUE1 command is used to obtain a by-residue RMS
deviation on the currently specified RMS atoms in the currently
specified RMS zone. If no \verb1RATOMS1 and \verb1RZONE1 commands have
been used, the atoms and zones used for the fitting will be used.

The \verb1RESIDUE1 command may be followed by an optional filename
parameter in which case output is directed to the specified file. If
the file cannot be opened or a filename is not specified, output
appears on the screen. If the first character of the filename is a
pipe character ($\mid$), then the results will be piped into the
specified program. For example:
\begin{verbatim}
   RESIDUE |less
\end{verbatim}
will cause the results to be displayed on the screen using the
\verb1less1 pager program.

If the distance cutoff is set then residues fully outside the distance cutoff 
are flagged with \verb1**1 and residues partially outside the distance cutoff 
are flagged with \verb1*1 (see Section~\ref{sec:distcutoff})


The related command, \verb1PAIRDIST1 prints the pairwise distances between 
equivalent atom pairs in the reference and mobile structures. \verb1PAIRDIST1 
has the same syntax as \verb1RESIDUE1. If the distance cutoff is set then 
residues outside the distance cutoff are flagged with \verb1*1.

\subsection{Outputting Fit Zones As An Alignment}
%-----------------------------------------------

As of \pf\ V3.0, it is possible to output the equivalenced regions found by 
iterative fitting as an alignment using the \verb1PRINTALIGN1 command. The 
default output is a (user-friendly) pairwise alignment with the reference and 
mobile sequences printed as pairs of 60-character wide lines. The optional 
\verb1FASTA1 and \verb1PIR1 parameters set the printout to (machine-friendly) FASTA or PIR formatting 
for the chain names and sequences.

Alignments can be exported to a text file using the \verb1PRINTALIGN1 command. \pf\ V3.0 can read PIR formatted files for assigning zones.


{\bfseries Note:} For a set of fit zones to be converted into an alignment, 
the fitting zones must occur sequentially along the protein chain.
Additionally, the fit zones cannot overlap.
In other words, to obtain a sequence alignment the fitting zones must be in 
sequence. 




%%%%%%%%%%%%%%%%%%%%%%%%%%%%%%%%%%%%%%%%%%%%%%%%%%%%%%%%%%%%%%%%%%%%%%
\section{Modifying the Fit}

\label{sec:modfit} 
The commands described in this section make use of the temperature factor 
column as a 'flag' and are therefore not compatible with the \verb1BZONE1 
command (Section~\ref{sec:bzone}).

Normally, no weighting is applied during the fitting i.e.\ all atoms
are weighted equally. The \verb1WEIGHT1 command causes the fitting to
be weighted by the mean of the B-values in the equivalent
atoms. Normally, you wouldn't use this with real B-values, but with
some other weight parameter (e.g.\ SSAP scores).

The \verb1BWEIGHT1 command weights the fitting by the inverse of the
mean of the B-values in the equivalent atoms. This is useful for
genuine weighting by B-values (i.e.\ the mobile set atoms will be less
heavily weighted).

The \verb1NOWEIGHT1 command switches off weighting.

Atoms can also be removed from consideration in the fitting and RMS
deviation calculations using temperature factors as a cutoff. The
\verb1BVALUE1 command allows you to specify a B-value cutoff and any
atoms with B-values greater than this value will be \emph{ignored
completely} in both the fitting and RMS deviation calculations. The
B-value may not be higher than this value in either the reference set
or the mobile set. For example, if you specify $10$, then atoms with
B-values greater than $10$ will be ignored.

By specifying a negative value for \verb1BVALUE1, you require that any
atoms with B-values less than the absolute value you specify will be
ignored. For example, if you specify $-10$, then atoms with B-values
less than $10$ will be ignored.

The value may be followed by an optional \verb1REF1 or \verb1MOB1
parameter which restricts checking of B-values to the specified
structure.



%%%%%%%%%%%%%%%%%%%%%%%%%%%%%%%%%%%%%%%%%%%%%%%%%%%%%%%%%%%%%%%%%%%%%%
\section{Script Files}

\label{sec:script} 

While it is possible to run a script from the unix command line using a redirection operator ($<$) or pipe ($\mid$), there are occasions when this is problematic such as when running \pf\ from within another application. It is possible to use a command line flag to run a script file.

For example, a script file can be run using either a command line flag:
\begin{verbatim}
   profit -f myscriptfile.txt -h reference.pdb mobile.pdb
\end{verbatim}
By using the redirection operator:
\begin{verbatim}
   profit -h reference.pdb mobile.pdb < myscriptfile.txt 
\end{verbatim}
Or by piping input from another program:
\begin{verbatim}
   cat myscriptfile.txt | profit -h reference.pdb mobile.pdb
\end{verbatim}
All three options produce identical outputs. 

It is also possible to run a script from within \pf\ using the SCRIPT command:
\begin{verbatim}
   SCRIPT myscriptfile.txt
\end{verbatim}

When a script file is run, messages indicating the start and end of the script are sent to stdout, if quiet mode is off. A comment marker (\verb1#1) at the beginning of a line will echo the line to stdout, a useful method for annotating an output file when running non-interactively.

Finally, it is possible to run a script from within a script using the \verb1SCRIPT1 command. \pf\ tracks the number of open/nested scripts and will allow up to 1000 nested scripts to be open. The assumption is that if over a thousand scripts are open then \pf\ has been sent into an infinite loop (for instance by having a script call itself). 


%%%%%%%%%%%%%%%%%%%%%%%%%%%%%%%%%%%%%%%%%%%%%%%%%%%%%%%%%%%%%%%%%%%%%%
\section{Miscellaneous Commands}

The \verb1RMS1 command may be used to reprint the RMS deviation over
the currently defined set of RMS zones and RMS atoms.


If you simply wish to calculate the RMSd between two or more structures 
without actually fitting them, defining fitting regions in the normal way then 
typing the \verb1NOFIT1 command (instead of the \verb1FIT1 command) 
will set up \pf\ to perform RMSd calculations but will not fit the structures. 
The \verb1RMS1 command can then be used to print the RMS deviation. 


As of \pf\ V3.0 it is possible to match symmetrical atoms automatically in 
amino acid sidechains (e.g.CD1 - CD2 and CE1 - CE2 of tyrosine) using the 
\verb1SYMMATOMS1 command. \verb1SYMMATOMS1 matches the charged oxygens and 
nitrogens on arginine, aspartate and glutamate residues and the delta and 
epsilon carbons of phenylalanine and tyrosine residues. It is also possible 
to match the nitrogen and oxygen atoms of the amide 
sidechains of asparagine and glutamine residues and the prochiral methyl 
groups of valine and leucine. Typing \verb1SYMMATOMS1 will display the pairs 
of atoms currently matched by \pf . Typing \verb1SYMMATOMS ON1 or 
\verb1SYMMATOMS OFF1 will turn symmetrical atom matching on or off.
Individual residue types, for example ASP, can be turned-on or off by 
typing SYMMATOMS ASP ON or SYMMATOMS ASP OFF, respectively. Alternatively, 
SYMMATOMS ALL ON will turn all atom pairs on. By default, the matching of 
symmetrical atoms is turned-off.


Any operating system command may be run from within \pf\ by
preceding it with a \verb1$1. The string following the \verb1$1 is
passed to the operating system exactly as given and is useful for
obtaining directory listings, typing, editing or copying files.




%%%%%%%%%%%%%%%%%%%%%%%%%%%%%%%%%%%%%%%%%%%%%%%%%%%%%%%%%%%%%%%%%%%%%%
\section{Command Summary}

\begin{description}

\item[\$ \emph{command}] Passes command to the operating system.

\item[\# \emph{comment}] Echoes comment to stdout.

\item[ALIGN [[WHOLE$\mid$*\mbox{]}$\mid$[\emph{zonespec} [APPEND\mbox{]}\mbox{]}\mbox{]}]
Performs Needleman and Wunsch sequence alignment on the
sequences of the two structures and derives zones from the equivalent
regions in the alignment. For multiple structure fitting, \verb1ALIGN1 
performs a pairwise alignments for the reference sequence and each mobile 
sequence.

   It will normally be necessary to use the ATOMS command to specify
   that only backbone or C-alpha atoms are included in the fitting
   calculations.

\item[ALLVSALL [\emph{filename}\mbox{]}] Performs an all versus all 
comparison of the mobile structures when fitting multiple structures.
Results are presented as tab-delimited text suitable for loading into
a spreadsheet. 

If the optional filename parameter is given, output is directed to
the specified file. If the file cannot be opened or a filename is
not specified, output appears on the screen. If the filename begins
with a pipe character ($\mid$), the results are piped into the
specified program.

\item[ATOMS \emph{atm[,atm]\ldots}] Specifies the atom subset to fit.

\item[BVALUE \emph{cutoff} [ REF$\mid$MOB\mbox{]}] Specify a B-value
cutoff. Any atoms with B-values greater than this value will be
ignored completely. A negative cutoff specifies that atoms with
B-values less than the absolute cutoff should be ignored. The optional
\verb1REF1 or \verb1MOB1 parameter restricts B-value checking to
the specified structure.

\item[BWEIGHT] Weight the fitting by the inverse of the mean of the
B-values in the equivalent atoms.

\item[BZONE] Sets fitting zones based on markers in the temperature factor (B-value) column

\item[CENTER [ OFF \mbox{]}] See \verb1CENTRE1.

\item[CENTRE [ OFF \mbox{]}] Cause the coordinates to be written (using
the \verb1WRITE1 or \verb1MWRITE1 commands), with the centre of
geometry located at the origin instead of in the same coordinate frame
as the reference set.

\item[DELRZONE \emph{zonespec}] Removes a zone specification to the list of
zones considered in RMS deviation calculation. \verb1DELRZONE *1 or
\verb1DELRZONE1 \verb1ALL1 deletes all RMS deviation calculation zones.

\item[DELZONE \emph{zonespec}] Removes a zone specification to the list of
zones considered in fitting. \verb1DELZONE *1 or \verb1DELZONE1 \verb1ALL1
removes all zone specifications.

\item[DISTCUTOFF [\emph{cutoff} $\mid$ ON $\mid$ OFF\mbox{]}]
Specifies a distance cutoff for RMSd calculations.

\item[FIT] Performs the actual fitting. Returns the RMS deviation over
the atoms included in the fit.

\item[GAPPEN \emph{val} [\emph{val}\mbox{]}] Specifies an integer gap 
penalty and a gap extension penalty for the sequence alignment performed 
by the \verb1ALIGN1 command. The default values for the gap penalty and gap 
extension penalty are 10 and 2 respectively.

\item[HEADER [ON $\mid$ OFF\mbox{]}] Include PDB header and trailer records when writing structures. By default, only the coordinate section of a file is output when a structure is written.

\item[HETATOMS] Read HETATM records with subsequent \verb1MOBILE1 and
\verb1REFERENCE1 commands.

\item[IGNOREMISSING] Ignore any atom mismatches and proceed with the
fitting. Such atoms are listed as warnings.

\item[ITERATE [ (\emph{limit} $\mid$ OFF) \mbox{]}] Switches on (or
off) iterative updating of the zones for fitting. The \verb1ITERATE1
command may be followed by an optional distance cutoff (default:
3.0\AA) or by the keyword `OFF' to switch off iterative zone
calculation.

\item[LIMIT (\emph{pos1 pos2} $\mid$ OFF)] Limits the range in an
alignment (from \verb1READALIGNMENT1) used to derive zones. 
\verb|LIMIT OFF| restores the default behaviour.

\item[MATRIX] Displays the centres of geometry, rotation matrix and
translation vector.

\item[MOBILE [ XMAS \mbox{]} \emph{\bfseries filename}] Reads a mobile
  PDB structure.  If a \% and number is added to the end of the
  filename then that model number will be read instead of the first.
  If compiled with XMAS support, then the XMAS keyword specifies that
  the input is in XMAS format.

\item[MULTI \emph{filename}] Reads a file of files containing a list
of structures for multiple fitting. Model numbers may be included in
those filenames.

\item[MULTREF [OFF\mbox{]}]
   Sets RMSd calculations to give values to the averaged reference rather 
   than the first mobile structure.

 \item[MWRITE [ \emph{ext}]] Write the results of multiple structure
   fitting. The structures are written back using the same filenames
   with which they were read, but with the extension changed to that
   specified. If no extension is given, then `.fit' is used. Note
   therefore, that you must have write permission to the directory
   from which the input files were read. If you specified a model
   number for the input files, this will be included in the output
   files (except for model 1).

\item[NFITTED] Reports the number of atom pairs fitted.

\item[NOFIT] Sets the fitted flag in profit allowing the user to calculate 
the RMSD on a structure without fitting.

\item[NOHETATOMS] Do not read HETATM records with subsequent
\verb1MOBILE1 and \verb1REFERENCE1 commands.

\item[NOIGNOREMISSING] Restore the default behaviour of issuing an
error message for any atom mismatches and halting the fitting
proceedure.

\item[NOWEIGHT] Normal, non-weighted fitting.

\item[NUMBER (RESIDUE$\mid$SEQUENTIAL)] Specifies whether zones are
based on residue numbers in the PDB file or on sequential numbering
(running through all chains).

\item[OCCRANK \emph{n}] Sets ProFit to read the \emph{n}th ranked highest 
occupancy atom position for alternative atom positions.

For structure files containing partial occupancies, lower occupancy atoms 
can be read using by setting the occupancy rank parameter to read alternative 
atom positions.

By default, OCCRANK is set to 1 and reads the highest ranked atom position, 
a setting of 2 will read the second most occupied position and a setting of 3 
will read the third most occupied position, etc.

\item[ORDERFIT]
   Performs a fit of all mobile structures to the reference structure. The 
   most similar structures are fitted first.


\item[PAIRDIST [ \emph{filename}\mbox{]}] Prints the pairwise distances between equivalent atom pairs.
If the first character of the (optional) filename is a pipe character
($\mid$), then the results will be piped into the specified
program. For example:
\begin{verbatim}
   PAIRDIST |less
\end{verbatim}
will cause the results to be displayed on the screen using the
\verb1less1 pager program.

\item[PRINTALIGN [FASTA$\mid$PIR\mbox{]} [ \emph{filename}\mbox{]}] 
   Prints current fitting zones as a sequence alignment.
   The default output is a (user-friendly) pairwise alignment with the 
   reference and mobile sequences printed as pairs of 60-character wide 
   lines.
   The optional \verb1FASTA1 and \verb1PIR1 parameters set the printout 
   to FASTA or PIR formatting.
   \pf\ can read PIR-formatted files using the \verb1READALIGN1 command.

\item[QUIET [ OFF \mbox{]}] Switches on (or off) quiet mode. In quiet
mode, warning messages are suppressed and progress of iterative zone
updating and multiple structure fitting is not reported.

\item[QUIT] Exits from the program.

\item[RATOMS \emph{atm[,atm]\ldots}] Specifies atoms over which to
calculate the RMS deviation. Fitting must already have been performed.

\item[READALIGNMENT \emph{filename}] Reads an alignment in PIR
sequence file format and sets zones based on that alignment.  Note
that when used with multiple structures, the first sequence \emph{must
appear twice} in the alignment file. This is because it is used as
both the first reference and mobile set.

\item[REFERENCE [ XMAS \mbox{]} \emph{\bfseries filename}] Reads a
  reference PDB structure.  If a \% and number is added to the end of
  the filename then that model number will be read instead of the
  first. If compiled with XMAS support, then the XMAS keyword
  specifies that the input is in XMAS format.

\item[RESIDUE [ \emph{filename}]] Gives a by-residue RMS deviation.
If the first character of the (optional) filename is a pipe character
($\mid$), then the results will be piped into the specified
program. For example:
\begin{verbatim}
   RESIDUE |less
\end{verbatim}
will cause the results to be displayed on the screen using the
\verb1less1 pager program.

\item[RMS] Recalculate the RMS deviation over the zones and atoms
currently defined with \verb1RZONE1 and \verb1RATOMS1.

\item[RZONE \emph{zonespec}] Adds a zone specification to the list of
zones considered in RMS deviation calculation. \verb1RZONE *1 or
\verb1RZONE1 \verb1CLEAR1 resets the zones for RMSD calculation to be
the same as that specified with the \verb1ZONE1 command.
   
\item[SCRIPT \emph{filename}] Executes a script file.

\item[SETCENTER \emph{residue}] See \verb1SETCENTRE1.

\item[SETCENTRE \emph{residue}] Specifies a single residue as the centre of fitting. Entering SETCENTRE CLEAR or SETCENTRE * will clear the centre residue.

   \item[SETREF [\emph{n}\mbox{]}]
   Sets the reference structure to the \emph{n}th mobile structure when fitting multiple structures.

   If no structure number is given then the reference is automatically set 
   by performing an all versus all comparison of the mobile structures then
   selecting the structure with the lowest overall RMSD to the other mobile 
   structures.

\item[STATUS[ \emph{filename}]] Reports current program status. 
   If the optional filename parameter is given, output is directed to
   the specified file. If the file cannot be opened or a filename is
   not specified, output appears on the screen. If the filename begins
   with a pipe character ($\mid$), the results are piped into the
   specified program.

\item[SYMMATOMS [[OFF$\mid$ON$\mid$ALL\mbox{]}$\mid$\emph{xxx} [OFF$\mid$ON\mbox{]}]

   where \emph{xxx} is a three-letter amino acid code.

   Enabes the auto-matching of symmetrical atoms (eg CD1 - CD2 and 
   CE1 - CE2 of tyrosine) in \pf\

   \verb1SYMMATOMS1 matches charged oxygens and nitrogens on arginine, 
   aspartate and glutamate residues and the delta and epsilon carbons of 
   phenylalanine and tyrosine residues.

   It is also possible to match the nitrogen and oxygen atoms of the amide 
   sidechains of asparagine and glutamine residues and the prochiral methyl 
   groups of valine and leucine.

   Typing \verb1SYMMATOMS1 will display the pairs of atoms currently matched 
   by \pf. Typing \verb1SYMMATOMS1 \verb1ON1 or \verb1SYMMATOMS1 \verb1OFF1 
   will turn symmetrical atom matching on or off.

   Individual residue types, for example ASP, can be turned-on or off by 
   typing \verb1SYMMATOMS1 \verb1ASP1 \verb1ON1 or \verb1SYMATM1 \verb1ASP1 
   \verb1OFF1, respectively. Alternatively, \verb1SYMMATOMS1 \verb1ALL1 
   \verb1ON1 will turn all atom pairs on.

   By default, the the matching of symmetrical atoms is turned-off


\item[TRIMZONES]
   This command is used primarily with fitting zones derived using \verb1ALIGN1. 
   With pairwise alignments, the lengths of the aligned regions may vary and
   there may be gaps in the alignments from one structure to another. The 
   \verb1TRIMZONES1 command trims the ends of the aligned zones and adds gaps 
   allowing for a like versus like comparison by using fitting zones that are
   common to all the structures.

   \verb1TRIMZONES1 is automatically called by the \verb1ALLVSALL1 and 
   \verb1SETREF1 commands. 
   This command is only used with multiple structures.

\item[WEIGHT] Weight the fitting by the mean of the B-values in the
equivalent atoms.

\item[WRITE [ REFerence \mbox{]} \emph{\bfseries filename}] Writes the fitted structure
to a PDB file. If the first character of the filename is a pipe
character ($\mid$), then the results will be piped into the specified
program. For example:
\begin{verbatim}
   WRITE |less
\end{verbatim}
will cause the coordinates to be displayed on the screen using the
\verb1less1 pager program.

If the \verb1REFERENCE1 keyword is given (only the letters `REF' are
required), then the reference set will be written. This is used in
conjunction with the \verb1CENTRE1 command.

\item[WTAVERAGE [ ON$\mid$OFF \mbox{]}]

   Sets the weighting system for the averaged reference structure to the 
   default weighting system where the change in the coordinates of the 
   reference structure is inversely proportional to the number of mobile 
   strucures. The weighted averaging scheme was introduced to lower the 
   effect that outlying structures have on the averaged reference.
   (Default: ON)

   The alternative weighting scheme sets the coordinates of the 
   reference stucture to the average of the reference and the mobile 
   structures. This was the scheme used by ProFit prior to version 3.0. 
   (\verb1WTAVERAGE OFF1)

\item[ZONE \emph{zonespec}] Adds a zone specification to the list of
zones considered in fitting. \verb1ZONE *1 or \verb1ZONE1 \verb1CLEAR1
removes all zone specifications.
\end{description}

%%%%%%%%%%%%%%%%%%%%%%%%%%%%%%%%%%%%%%%%%%%%%%%%%%%%%%%%%%%%%%%%%%%%%%
\section{Copyright}

Please note that the program is called \pf\ --- not PROFIT, Profit or
profit; this attempts to avoid confusion with the threading program
known as PROFIT. \pf\ was written first and released to the public
around the same time. 

\pf\ is pronounced as it is written, i.e.\ `pro' (as in `protein') then
`fit' (not `profit' as in `make lots of money'!

\pf\ was initially written by Dr Andrew C.R. Martin while self-employed and 
trading as {\sffamily SciTech Software}. 
Enhancements have been made since at UCL and at the University of Reading. 
Addition of iteration and multiple fitting was sponsored by Inpharmatica, Ltd. 
Enhancements in V2.6 and V3.0 were made possible by a Tools and Resources 
grant from the BBSRC.



\begin{quotation}
This program is not in the public domain.

It may not be copied or made available to third parties, but may be
freely used by non-profit-making organisations and commercial
companies who have obtained it directly from the author or by FTP or
HTTP from the author's web sites.

If you did not register the program via the web site, you are
requested to send EMail to the author to say that you are using this
code so that you may be informed of future updates.

The code may not be made available on other FTP or Web sites without
express permission from the author.

The code may be modified as required, but any modifications must be
documented so that the person responsible can be identified. If
someone else breaks this code, the author doesn't want to be blamed
for code that does not work! You may not distribute any modifications,
but are encouraged to send them to the author so that they may be
incorporated into future versions of the code.

While the compiled \pf\ program may be used by commercial companies,
it may not be sold commercially or included as part of a commercial
product. The source code or any derivative works may not be sold
commercially or used for commercial purposes outside of \pf\ without
prior permission from the author.
\end{quotation}

\vspace{2em}

While this software is provided ``as is'' and free of charge, I
do appreciate hearing from people who use it and find it useful. An
EMail or a postcard would be nice.

If you find \pf\ useful, please tell your colleagues about it. Please
\emph{do not} pass copies of \pf\ on to them directly; ask them to
obtain it \emph{via} my World Wide Web page
(\verb1http://www.bioinf.org.uk/software/profit/1)

%%%%%%%%%%%%%%%%%%%%%%%%%%%%%%%%%%%%%%%%%%%%%%%%%%%%%%%%%%%%%%%%%%%%%%
\section{How do I Reference \pf?}

No paper has been published describing \pf\ itself since it is simply
a convenient program (I hope) to let you use a standard fitting
algorithm; consequently, it is a little difficult to reference. The
exact wording is up to you and dependent on the context, but I suggest
something similar to:

\begin{quotation}
Fitting was performed using the McLachlan algorithm (McLachlan, A.D.,
1982 ``Rapid Comparison of Protein Structres'', Acta Cryst A38,
871-873) as implemented in the program ProFit (Martin, A.C.R. and
Porter, C.T.,
\verb1http://www.bioinf.org.uk/software/profit/1) 
\end{quotation}

%%%%%%%%%%%%%%%%%%%%%%%%%%%%%%%%%%%%%%%%%%%%%%%%%%%%%%%%%%%%%%%%%%%%%%
\section{Acknowledgements}

Inpharmatica Ltd. are acknowledged for funding development of V2.0 of ProFit. 
The BBSRC are acknowledged for funding development of V2.6 and V3.0 of ProFit.

\end{document}
