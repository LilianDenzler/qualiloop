%%%%%%%%%%%%%%%%%%%%%%%%%%%%%%%%%%%%%%%%%%%%%%%%%%%%%%%%%%%%%%%%%%%%%%%%%%
%  Revision History
%  ================
%  20.09.94 Original    By: ACRM
%  21.09.94 Added IGNORE
%  22.09.94 Added to abstract, also data preparation and running programs
%  30.09.94 A couple of corrections and added installation and xpar/xtop
%  25.10.94 Added RESIDUE potential
%  10.11.94 Added notes on topology data file and made to V1.2
%  14.11.94 Added pdborder to command line example to generate coords
%  22.05.95 Added RELAX stuff for V1.4
%
%%%%%%%%%%%%%%%%%%%%%%%%%%%%%%%%%%%%%%%%%%%%%%%%%%%%%%%%%%%%%%%%%%%%%%%%%%
\documentclass[12pt]{article}
\usepackage{a4}

\newcommand{\ecalc}{\mbox{\bf ECalc}}
\newcommand{\ec}{\mbox{\bf EC}}
\newcommand{\csearch}{\mbox{\bf CSearch}}
\newcommand{\bioplib}{\mbox{\bf BiopLib}}
\newcommand{\degree}{\mbox{${}^\circ$}}
\begin{document}

%%%%%%%%%%%%%%%%%%%%%%%%%%%%%%%%%%%%%%%%%%%%%%%%%%%%%%%%%%%%%%%%%%%%%%%%%%
\begin{center}
{\Large\bf ECalc V1.5.1}
\large

{\bf A protein energy calculation program.}
\vspace{1em}

(c) 1994-2021, Prof.\ Andrew C.\ R.\ Martin,\\
Biomolecular Structure and Modelling Unit,\\
Department of Biochemistry and Molecular Biology,\\
University College London,\\
Gower Street,\\
London WC1E 6BT\\
\vspace{1em}

22nd May, 1995
\vspace{2em}

\end{center}

%%%%%%%%%%%%%%%%%%%%%%%%%%%%%%%%%%%%%%%%%%%%%%%%%%%%%%%%%%%%%%%%%%%%%%%%%%
\begin{abstract}
   \ecalc\ is a program designed to calculate the energy of a protein. It
may also be used to calculate the energies of a number of conformations 
of a region within a protein. The calculations may be restricted to
the region of interest (though non-bonded contacts with other parts of
the protein will also be considered). In addition, other parts of the
protein may be specified which should be excluded from consideration
in the non-bonded contact calculations. Parts of the potential may be
switched on or off and scaled with respect to one another. This
allows, for example, the simple calculation of just the van der Waals
repulsion or just the bond energy.

   The program may be run using a control file (as described below),
or using an X-windows interface (\ec) on machines equipped with the
Tcl/Tk package. This interface builds a control file and spawns the
normal \ecalc\ program to perform the calculations.
\end{abstract}


%%%%%%%%%%%%%%%%%%%%%%%%%%%%%%%%%%%%%%%%%%%%%%%%%%%%%%%%%%%%%%%%%%%%%%%%%%
\section{Installation}


\subsection{Unpacking}
\ecalc\ is supplied as a gzipped tar file. To unpack, go to the directory
in which you want the \ecalc\ directory to be installed and issue the
command:
\begin{verbatim}
   zcat ecalc.tar.gz | tar -xvf -
\end{verbatim}

This assumes that {\tt ecalc.tar.gz} is in your current directory (if not,
you must supply the full path) and that {\tt zcat} is the GNU/gzip 
version. If not, you may have to include the full path to the GNU 
{\tt zcat} program.



\subsection{Compiling}
To compile \ecalc, you must already have installed the \bioplib\
library. Assuming this has been done, change to the {\tt src}
subdirectory of the new {\tt ecalc} directory and edit the supplied
{\tt makefile}. Modify {\tt INCFLAGS} and {\tt LIBFLAGS} to reflect
the directory in which you have installed \bioplib. Modify {\tt CC}
and {\tt LINK} for your C compiler. You only need to alter {\tt PROTO}
if you wish to regenerate prototype files. This will only be necessary
if you have modified any of the routines.

Having modified the {\tt makefile} as required, type {\tt make} to
compile the program. Move the resulting {\tt ecalc} executable to
somewhere in your path (e.g.\ {\tt /usr/local/bin}).

Finally, type {\tt make clean} to remove the object files.



\subsection{Environment variables}
One environment variable should be set to specify the directory
containing the residue topology and parameter data files. Therefore,
if you are using a csh-like shell, you should add the following to
your {\tt .cshrc} file:
\begin{verbatim}
   setenv ECALCDATA '/usr/local/ecalc/data'
\end{verbatim}
where {\tt /usr/local/ecalc/data} is replaced by the name of the
directory in which your \ecalc\ datafiles are stored. If you are using
an sh-like shell, add the following to your {\tt .profile} file:
\begin{verbatim}
   export ECALCDATA='/usr/local/ecalc/data'
\end{verbatim}



\subsection{Customising for your local setup}
You must now customise the file {\tt local.tcl} for your local system.
Change to the data directory using the command:
\begin{verbatim}
   cd $ECALCDATA
\end{verbatim}

Now edit the file {\tt local.tcl}. Find the line which starts {\tt set
ecalc} and change the path to reflect the location of you \ecalc\
executable. For example:
\begin{verbatim}
   set ecalc "/usr/local/bin/ecalc"
\end{verbatim}

Find the line which starts {\tt set auto\_path} and modify the
directory specification to reflect your main \ecalc\ directory which
contains {\tt ecalc.tcl} and the other Tcl files.
\begin{verbatim}
   set auto_path "/usr/local/ecalc $auto_path"
\end{verbatim}

You may also modify the colours to be used for the buttons and the
maximum number of disulphides, zones and ignores if required.

Now move to the main \ecalc\ directory. Depending on the version of
Tcl/Tk you have installed, you will require either the {\tt
tclIndex.old} or {\tt tclIndex.new} file. Copy the appropriate version
to {\tt tclIndex}. If you have the wrong version, the ecalc interface
will exit with a message like: {\em format error in library index}.

Finally, make a symbolic link from somewhere in your path to the main
\ecalc\ Tcl file. For example:
\begin{verbatim}
   ln -s /usr/local/ecalc/ecalc.tcl /usr/local/bin/ec
\end{verbatim}



%%%%%%%%%%%%%%%%%%%%%%%%%%%%%%%%%%%%%%%%%%%%%%%%%%%%%%%%%%%%%%%%%%%%%%%%%%
\section{Utilities}
\label{sec:util}

Two utilities are provided to help convert XPLOR (CHARMM) parameter and
topology files to \ecalc\ format. These programs are written in Perl
and require that you have the Perl interpreter installed on your
system.

Unfortunately, while the programs will produce valid data files for
ECalc, the conversion is not complete; some hand editing will also be
necessary. This is because 
\begin{enumerate}
\item \ecalc\ requires all atom parameter data to be provided in the
parameter file while XPLOR has the atom masses and radii in the
topology file,
\item The XPLOR topology file does not contain atom exclusion data.
\end{enumerate}

Both conversion utilities are run in the same way:
\begin{verbatim}
   xpar.perl params.xplor   >params.ecalc
   xtop.perl topology.xplor >topology.ecalc
\end{verbatim}
You should substitute appropriate filenames for the {\tt .xplor} and
{\tt .ecalc} files sepcified here.

In addition the atom ordering in the residue topology file is critical
if the program is to be used for conformation screening. The ordering
must match that used in CSearch. i.e.\ N, H, CA, C, O, sidechain.


%%%%%%%%%%%%%%%%%%%%%%%%%%%%%%%%%%%%%%%%%%%%%%%%%%%%%%%%%%%%%%%%%%%%%%%%%%
\section{Preparing input data}
\label{sec:data}

Generally, \ecalc\ will be run to follow on from a \csearch\
(CHARMM-free CONGEN) run, in which case all the necessary files will
have been prepared already. The basic input files are a PDB file with
hydrogens and CHARMM-style NTER and CTER residues\footnote{All future
references to a PDB file will imply a PDB file with hydrogens, NTER
and CTER residues added.} and, optionally, a conformations file in
\csearch\ format.

A suitable PDB file may be prepared using the following Unix command
line:
\begin{verbatim}
   pdbhadd -c xxxx.pdb | pdbcter -c | renumpdb | pdborder -c > xxxx.pdh
\end{verbatim}
where {\tt xxxx.pdb} is a normal PDB file and {\tt xxxx.pdh} has had the
hydrogens, NTER and CTER residues added. The {\tt pdbhadd -c} command
adds hydrogens and an NTER residue, the {\tt pdbcter -c} command adds 
C-terminal oxygens and a CTER residue while {\tt renumpdb} renumbers
the chains sequentially.

%%%%%%%%%%%%%%%%%%%%%%%%%%%%%%%%%%%%%%%%%%%%%%%%%%%%%%%%%%%%%%%%%%%%%%%%%%
\section{Running the programs}

\subsection{EC}
\ec\ is the X-windows graphical interface for \ecalc. It is run by
typing {\tt ec} at the command line prompt.

\subsection{ECalc}
The command syntax for \ecalc\ is:
\begin{verbatim}
   ecalc [-p (xxxx.pdh | -)] [(control.dat | -)]
\end{verbatim}
where {\tt xxxx.pdh} is a PDB file (see Section~\ref{sec:data}) and {\tt
control.dat} is a control file (see Section~\ref{sec:control}). In
either case, the filename may be replaced by a {\tt -}, in which case,
the appropriate input will be expected on standard input.

If \ecalc\ is run with a PDB file, but no control file, the
full-potential energy of the whole structure will be calculated and
displayed. Normally, \ecalc\ will be run with a control file and the
PDB filename will be specified within the control file.


%%%%%%%%%%%%%%%%%%%%%%%%%%%%%%%%%%%%%%%%%%%%%%%%%%%%%%%%%%%%%%%%%%%%%%%%%%
\section{The control file}
\label{sec:control}

\subsection{PDBFILE  {\em filename}}
Specifies the PDB file to be used for energy calculation. If not
specified in the file, the PDB file may be specified on the command
line using the {\tt -p} option.

\subsection{CONFFILE {\em filename}}
Specifies an optional conformation file. If specified, the energy of
each conformation in the context of the PDB file will be calculated.

\subsection{OUTFILE  {\em filename}}
Specifies an optional output file. If not specified, output goes to
the standrad output stream.

\subsection{POTENTIAL}
If the {\bf POTENTIAL} keyword is not specified, the standard full
potential will be calculated. If the keyword is given, it is followed
by one or more of the commands given in Table~\ref{tab:pot} {\em on
separate lines}. After all potential options have been specified, the 
{\bf END} keyword must be given. 

\begin{table}[h]
\begin{center}
\begin{tabular}{ll}
BONDS [{\em scale}]          & Calculate bond energy                  \\
ANGLES [{\em scale}]         & Calculate angle energy                 \\
TORSIONS [{\em scale}]       & Calculate torsion angle energy         \\
IMPROPERS [{\em scale}]      & Calculate improper torsion angle energy\\
HBONDS [{\em scale}]         & Calculate hydrogen bond energy         \\
VDWA [{\em scale}]           & Calculate van der Waals attraction     \\
VDWR [{\em scale}]           & Calculate van der Waals repulsion      \\
ELECTROSTATICS [{\em scale}] & Calculate electrostatic energy         \\
RESIDUE [{\em scale}]        & Calculate an empirical residue energy  \\
\end{tabular}
\end{center}
\caption{\label{tab:pot} Keywords specifying parts of the potential to
be included. In each case, the scale parameter is optional and, if not
specified, will default to $1.0$}
\end{table}

\subsection{DISPLAY}
The {\bf DISPLAY} command controls which components of the energy are
displayed in the output. By default, only the total energy is
displayed. The command is followed by keywords from
Table~\ref{tab:pot}, each of which must be given on a separate line
and without the {\em scale\/} parameter.
As with the {\bf POTENTIAL} command, the options are ended with the
{\bf END} command.

\subsection{CACHE $n$}
At the end of the run, the conformation numbers and energies of the
lowest energy conformations are displayed. The {\bf CACHE} command is
used to control how many conformations should be displayed. The
default is 1 conformation. A value of 0 is also legal.

\subsection{GRIDCUT $n$}
\label{sub:gridcut}
This is a very important optimisation parameter. In order to speed the
calculation of non-bonded interactions, a `contact grid' is
constructed at the beginning of the run using the coordinates in the
parent structure. The grid is not recalculated at any time. This means
that non-bonded atom contacts are only considered between atoms in the
grid, thus speeding the calculations for each conformation. However,
the atoms in each conformation are in different positions, so the grid
must be larger than the non-bonded cutoff distance by an amount at
least equal to the greatest movement of any one atom. 

The utility {\tt spancg} reports the greatest movement from the reference
position for any atom in a CG file. The value reported by this program
should be added to the non-bonded cutoff distance and used for the
{\bf GRIDCUT} parameter

See also the REGRID command (Section~\ref{sub:regrid}).

\subsection{NONBONDCUT $n$}
Cutoff to be used when calculating van der Waals and electrostatic
energies. Any atom pairs separated by more than this distance will not
be considered in calculating the energy. Defaults to 8.0\AA.

\subsection{CONSTDIELECTRIC}
Use a constant dielectric when calculating the electrostatic energy. 
The value of the dielectric constant is specified using the 
{\bf ETA} keyword. i.e. $$ E_{el} = C_{el}\frac{q_1 q_2}{d\eta} $$

\subsection{DISTDIELECTRIC}
Use a distance dependent dielectric when calculating the electrostatic
energy. i.e. $$ E_{el} = C_{el}\frac{q_1 q_2}{d^2} $$

\subsection{ETA $n$}
Specifies the dielectric constant to be used when a constant
dielectric is selected. Default = 50.0

\subsection{CUTOFFHB $n$}
Specifies the distance above which hydrogen bond energy should not be
calculated. Atom pairs separated by more than this distance will be
ignored. Default: 5.0\AA.

\subsection{CUTONHB $n$}
Specifies a distance at which smoothing of the hydrogen bond energy
calculations should begin. This prevents discontinuities as a result
of the hydrogen bond distance cutoff. {\bf CUTONHB} must be $\le$ {\bf
CUTOFFHB}. Default: 4.0\AA.

\subsection{CUTOFFANG $n$}
Specifies an angle below which hydrogen bond energy should not be
calculated. If the angle subtended at the hydrogen by the acceptor and
the hydrogen's antecedent is less than this value, the hydrogen bond
will be ignored. Default: 90\degree.

\subsection{CUTONANG $n$}
Specifies an angle at which smoothing of the hydrogen bond energy
calculations should begin. This prevents discontinuities as a result
of the hydrogen bond angle cutoff. {\bf CUTONANG} must be $\ge$ {\bf
CUTOFFANG}. Default: 90\degree.

\subsection{SHOWPARAMS}
Causes parameters read from the parameter file to be displayed.

\subsection{SHOWRTOP}
Causes the residue topology read from the residue topology file to be
displayed.

\subsection{SHOWTIMINGS}
Causes timings to be displayed after calculation of the close contacts
grid and each conformation.

\subsection{PARAMFILE {\em filename}}
Specifies the parameter file. Defaults to {\tt params.dat}.

\subsection{RTOPFILE {\em filename}}
Specifies the residue topology file. Defaults to {\tt topology.dat}.

\subsection{DEBUG}
Switches on certain debugging options. Currently only the routine
name display with each error message is switched on. This gives a
simulated trace-back of where the error occurred.

\subsection{DISULPHIDES (OFF $\mid$ {\em firstres lastres})}
By default, disulphides are found by a distance search; cysteine
S$\gamma$ atoms separated by less than 2.236\AA\ are considered to be
bonded. If the {\bf DISULPHIDES} keyword is followed by the word {\bf
OFF}, no search for disulphides will be performed. Alternatively, the
{\bf DISULPHIDE} command may be followed by residue pair
specifications for disulphides to be specified manually. This also
causes the automatic search for disulphides to be switched off.
Residues are specified using the chain name (if present) followed
immediately by the residue number (e.g.\ {\tt L24}). Insertions in the
numbering are not allowed; numbering is consecutive starting with the
NTER residue being residue number 1.


\subsection{REGRID $n$}
\label{sub:regrid}
By default, a close contacts grid is only calculated once, before any
energies are calculated. If {\bf REGRID} is specified with a parameter
of 1, the grid will be recalculated before each energy is evaluated.
Other values may also be used, but this must be done with caution; for
example, you must know that, for example, every 10 conformations fall
into a cluster in space. See the GRIDCUT command
(Section~\ref{sub:gridcut}) for further details on the close contacts
grid.


\subsection{ZONE {\em firstres lastres}}
Specifies a zone over which calculations should be performed. More
than one {\bf ZONE} command may be specified. Bonded topology (bonds,
angles, torsions and impropers) which does not include at least one
atom in a zone is discarded. Non-bonded contacts (van der Waals,
electrostatics and hydrogen bonds) are considered between atoms in the
specified zones and atoms outside the zones.

Residues are specified using the chain name (if present) followed
immediately by the residue number (e.g.\ {\tt L24}). Insertions in the
numbering are not allowed; numbering is consecutive starting with the
NTER residue being residue number 1.

When applying zones to regions included in a conformation file, you
should make the zones 1 residue wider on either side than the zone
covered by the conformations. This is because of the way in which
splicing between conformation and framework is performed by CSearch.

\subsection{IGNORE {\em firstres lastres} [SIDECHAIN]}
Specifies a zone to be ignored when calculating non-bond contacts. More
thans one {\bf IGNORE} command may be specified. Any non-bonded 
interactions (van der Waals, electrostatics and hydrogen bonds) which
involve atoms in the specified range are ignored.

If the optional {\bf SIDECHAIN} qualifier is specified, only sidechain
atoms (from the $\gamma$ atom out) are ignored.

Residues are specified using the chain name (if present) followed
immediately by the residue number (e.g.\ {\tt L24}). Insertions in the
numbering are not allowed; numbering is consecutive starting with the
NTER residue being residue number 1.

\subsection{RUN}
This command is provided only for interactive use and causes the
actual calculations to begin. And end-of-file also has the same effect.

\subsection{RELAX}
Performs a relaxation of the worst van der Waals contacts followed by
the SHAKE algorithm to regularise bond lengths.

\subsection{TOL [SHAKE$\mid$VDW] {\em value}}
Specifies the tolerence for the SHAKE algorithm (default: 0.001) or
the allowed fraction VDW clash (default: 0.5). Higher values for VDW
allow more atoms to clash without taking any action. The value is the
fraction of the optimum separation squared by which the atoms clash.

%%%%%%%%%%%%%%%%%%%%%%%%%%%%%%%%%%%%%%%%%%%%%%%%%%%%%%%%%%%%%%%%%%%%%%%%%%
\section{Future ideas}

Modify the grid recalculation calls to store the conformation used at grid
calculation time then, for each conformation, check the atom positions to see if
any atom has moved outside the current grid. If so recalculate the grid and
store the new conformation.



\end{document}
